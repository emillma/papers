\pagebreak
\section{Safeguarding Authorship Standards: Practical Measures}
Even though most people agree that legitimate authorship is essential, disputes and conflicts related to authorship can still arise \cite{pilebergCoauthorshipExpertsHave2022}.
In this section, we will discuss some practical measures that can be taken to safeguard authorship standards and solve authorship-related issues, both on an institutional and individual level.

\subsection{Ensuring a Clear Understanding of Authorship Standards}
The first practical step to ensure good authorship practices is ensuring everyone involved in research at an institution understands the authorship standards.
To this end, institutions should have clear and accessible guidelines on authorship standards, which should be communicated to all researchers through mandatory training sessions.
For example, the Norwegian University of Science and Technology (NTNU) has guidelines on this matter online, referencing the Vancouver Guidelines, and organizes courses for PhD students covering the matter \cite{ntnuCoauthorshipKnowledgeBase} \cite{ntnuCourseResearchEthics}.
A shared understanding of authorship standards will make discussions about authorship easier and more productive, but it does not ensure that conflict will not arise \cite{pilebergCoauthorshipExpertsHave2022}.
Discussing authorship with colleagues and supervisors can help researchers understand the standards and how they apply to their specific research, as it may vary between fields and institutions \cite{pilebergCoauthorshipExpertsHave2022}.

\subsection{Start the Conversation Early}
Solving an authorship conflict can be difficult \cite{faulkesResolvingAuthorshipDisputes2018}, which is why it is vital to start the conversation about authorship early in the research process \cite{pilebergCoauthorshipExpertsHave2022} \cite{albertHowHandleAuthorship2003}.
Establishing each person's role and expected contribution to the research alongside the authorship order can help prevent disputes from arising later on \cite{albertHowHandleAuthorship2003}.
It should preferably be done in a physical meeting \cite{albertHowHandleAuthorship2003}.

A written authorship agreement is recommended to avoid misunderstandings, making solving disputes later easier \cite{albertHowHandleAuthorship2003}.
The roles and contributions of the researchers might likely change during the research process \parencite[p.81]{ewartAuthorshipCoAuthoringCollaborating2023}, but having a written agreement makes the process of a potential renegotiation less ambiguous as the initial agreement can be used as a starting point for the discussion.

\subsection{Understand Each Author's Goal}
Lack of a shared goal can lead to authorship conflicts \cite{pilebergCoauthorshipExpertsHave2022}.
Suppose some researchers want to publish in a high-impact journal to fulfill some university criteria, while others are more interested in getting the research out quickly and building on it in future research.
In that case, they might have differing views on the expected workload and direction of the research.
Rejection of the paper by the journal can also lead to conflict if a fallback plan is not agreed upon beforehand.
Ensuring everyone is aware of each other's goals and how the collaboration is part of each person's research journey can help prevent conflict from arising \cite{pilebergCoauthorshipExpertsHave2022}.

\subsection{Solving Conflicts}
While good communication between the researchers can help prevent and solve disputes, more severe conflicts may require more substantial measures \parencite[p.82]{ewartAuthorshipCoAuthoringCollaborating2023}.
A helpful tool to solve a locked conflict is to recruit a third party as a mediator to get an impartial view on the issue \cite{pilebergCoauthorshipExpertsHave2022}.
Possible third parties include shared colleagues, supervisors, the head of the department, or the institution's Ethics Committee \cite{ntnuEthicsPortalNTNU}.
Unsolved disputes can block the progress of the research and can lead to a hostile working environment.
They will eventually have to be solved if the research is to be published, so it is better to solve them when they arise, even if it is uncomfortable.

\subsection{Handeling Misconduct}
While disputes can arise without malicious intent or unethical behavior, it is important to be aware of the possibility of misconduct.
If the ethical guidelines for research are not followed, it is considered research misconduct and should be reported \cite{albertHowHandleAuthorship2003}.
A relevant example of author-related ethical misconduct is when work is submitted to a journal without the final approval of all the authors or with contributing authors left out.
Reporting misconduct can be an uncomfortable experience, but failing to do so will indirectly contribute to an unethical work environment \cite{albertHowHandleAuthorship2003}.






