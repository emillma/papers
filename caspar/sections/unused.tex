\subsection{Expression Mapping}
A key part on sympolic computing is common subexpression elimination.
Consider the following example expression:
\begin{align}
     & \mathbf{x} = \begin{bmatrix}x_0 \\ x_1\end{bmatrix}  \\
     & \mathbf{y} =  \begin{bmatrix}y_0 \\ y_1\end{bmatrix}
    =\frac{\begin{bmatrix}\sin(x_0) & \sin(x_0)+x_1^2\end{bmatrix}}{\left\Vert \mathbf{x} \right\Vert}^T
    =  \begin{bmatrix}
           \nicefrac{\sin x_0}{\sqrt{x_0^2+x_1^2}} \\
           \nicefrac{\sin x_0+x_1^1}{\sqrt{x_0^2+x_1^2}}
       \end{bmatrix}
\end{align}
Writing out the full expression for each output element would result in redundant calculations.
By using common subexpression elimination, we store the intermediate results in temporary variables and reuse them when needed.

\begin{align*}
    r_0 = & \sin x_0            &  & = \sin x_0                                     \\
    r_1 = & x_1^2               &  & =x_1^2                                         \\
    r_2 = & r_0+r_1             &  & =\sin x_0+x_1^2                                \\
    r_3 = & x_0^2               &  & =x_0^2                                         \\
    r_4 = & r_1+r_3             &  & =x_0^2+x_1^2                                   \\
    r_5 = & \sqrt{r_4}          &  & =\sqrt{x_0^2+x_1^2}                            \\
    y_0 = & \nicefrac{r_0}{r_5} &  & =\nicefrac{\sin x_0}{\sqrt{x_0^2+x_1^2}}       \\
    y_1 = & \nicefrac{r_2}{r_5} &  & =\nicefrac{\sin x_0+x_1^2}{\sqrt{x_0^2+x_1^2}} \\
\end{align*}


In order to get better performance out of the GPU, we need to be aware of memory access patterns.
While using the data stored in $x$ directly would work, every time we access the data, we would have to fetch it from global memory.
\footnote{L2 cache partially solves this but is still slower than L1.}
To avoid this, we can store the data in registers as illustrated in (\ref{store_a}) and (\ref{store_b}).
When working on the GPU we can take advantage of special functions that are optimized for the GPU, such as \texttt{rnorm2d} and \texttt{fma}.

\begin{align}
    r_0 = & x_0                   \label{store_a}                                                      \\
    r_1 = & x_1                   \label{store_b}                                                      \\
    r_2 = & \sin r_0                              &  & = \sin x_0                                      \\
    r_3 = & fma(r_1, r_1, r_2)                    &  & =\sin x_0+x_1^2                                 \\
    r_0 = & rnorm2d(r_0, r_1)                     &  & =\nicefrac{1}{\sqrt{x_0^2+x_1^2}}               \\
    r_1 = & r_2*r_0                               &  & =\nicefrac{\sin x_0}{\sqrt{x_0^2+x_1^2}}        \\
    r_0 = & r_3*r_0                               &  & =\nicefrac{\sin x_0+x_1^2}{\sqrt{x_0^2+x_1^2}}  \\
    y_0 = & r_1                                   &  & = \nicefrac{\sin x_0}{\sqrt{x_0^2+x_1^2}}       \\
    y_1 = & r_0                                   &  & = \nicefrac{\sin x_0+x_1^2}{\sqrt{x_0^2+x_1^2}} \\
\end{align}

\subsection{Expression Ordering}
Each thread on the GPU has a finite set of 32-bit registers available.
If a kernel need more registers than whats available, local memory is used.
Ther is called register spilling an has a negative on performance, as local memory is as slow as global memory.
To minimize the amount of register spilling we want to find an optimal expression ordering.
This problem is called the bandwidth minimization problem and is NP-hard.

\begin{align*}
    r_0 = & x_0                                                                    \\
    r_1 = & \sin r_0           &  & = \sin x_0                                     \\
    r_2 = & x_1                                                                    \\
    r_0 = & rnorm2d(r_0, r_1)  &  & =\nicefrac{1}{\sqrt{x_0^2+x_1^2}}              \\
    r_2 = & fma(r_2, r_2, r_1) &  & =\sin x_0+x_1^2                                \\
    r_1 = & r_1*r_0            &  & =\nicefrac{\sin x_0}{\sqrt{x_0^2+x_1^2}}       \\
    y_0 = & r_1                &  & = \nicefrac{\sin x_0}{\sqrt{x_0^2+x_1^2}}      \\
    r_0 = & r_2*r_0            &  & =\nicefrac{\sin x_0+2x_1}{\sqrt{x_0^2+x_1^2}}  \\
    y_1 = & r_0                &  & = \nicefrac{\sin x_0+2x_1}{\sqrt{x_0^2+x_1^2}} \\
\end{align*}