\section{Conclusion and Future Work}
This paper presents a new portable sensor rig that makes the process of collecting high-quality datasets much easier.
The platform is fully self-contained, and its ease of use has already allowed multiple researchers to use it for their data collection.
The sensor rig is designed to be reproducible, and all the design files are available on request.
With the new datasets we have made available, we illustrate the potential of color polarization cameras in the maritime domain, as they reveal new relevant information about the scene.
With reflected light being polarized, the cameras make the water surface more visible and can be used to remove reflections and glare.
We believe the sensor rig and the data it captures can contribute to the development of new \gls{sitaw} systems for \glspl{asv}.

We intend to use the sensor rig frequently to collect numerous maritime datasets, covering different weather and lighting conditions and capturing various scenes from land and moving vessels.
With access to \gls{gnss} and \gls{imu} data from the sensor rig, we intend to mount similar sensors to other moving vessels to collect datasets where we know each vessel's position.
This will allow us to project each vessel's position into the video frames and create datasets to train and validate systems for detecting and tracking marine vessels.
While the sensors and components on the sensor rig are synchronized, the intrinsics and extrinsics of the sensors still need to be calibrated and validated, and we will add the parameters to the dataset in the future. Our goal with the sensor rig is to be a leading provider of high-quality maritime datasets to the research community for the development of \glspl{asv}.

\pagebreak
\section{Acknowledgements}
The Research Council of Norway funds this research through the SFI ATOSHIP project (project number 309230).
Emil Martens carried out the research and writing of this paper, with valuable supervision provided by Annette Stahl (main supervisor), Edmund Brekke, and Rudolf Mester.
