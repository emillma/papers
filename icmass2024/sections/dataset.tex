

\section{Dataset}
We have collected several sequences with the sensor rig to showcase the potential of the system.
The current dataset consists of one sequence where we filmed a moving \gls{usv} from land, one where we recorded data from a moving ship driving in the city canal and a two scenes where we walk along the river and film the water surface from different angles.
The dataset contains the raw data from the cameras together with metadata for each frame, raw data from the \gls{imu} and raw data from the \gls{gnss} receivers.
As we want to ensure quality before releasing calibration data, we currently provide a calibration sequence where we film a chess board while rotating the sensor rig, which can be used for calibration purposes.
A representative selection of images from the dataset can be found at the end of the paper, where we show the $S0$ image on the left and the polarized image or unpolaraized image on the right.

The dataset is available at \url{https://github.com/emillma/sensor-rig-datasets}.

\section{Conclusion and Future Work}
In this paper, we present a new portable sensor rig that significantly simplifies collecting sensor data in maritime environments.
The platform is fully self-contained, and its ease of use has allowed multiple researchers to use it for data collection.
We have also shown why polarization cameras could be beneficial in the maritime domain, as they reveal information about the scene, making the water surface more visible.
The datasets collected with the sensor rig illustrate this and show its potential as a tool for collecting data for the development of autonomous surface vehicles.

We intend to use the sensor rig frequently to collect many maritime datasets, covering different weather and lighting conditions and capturing different scenes both from land and from moving vessels.
With access to \gls{gnss} and \gls{imu} data from the sensor rig, we intend to mount similar sensors to moving vessels to collect datasets for detecting and tracking other vessels.
While the sensors and components on the sensor rig are synchronized, the intrinsics and extrinsics of the sensors still need to be calibrated and validated.

Our goal with the sensor rig is to be a leading provider of maritime datasets to the research community for the development of autonomous surface vehicles.





\begin{figure}[H]
    \begin{subfigure}[T]{.49\textwidth}
        \includegraphics[width=\textwidth]{figures/pictures/img_2790_s0.jpg}
    \end{subfigure} \hfill
    \begin{subfigure}[T]{.49\textwidth}
        \includegraphics[width=\textwidth]{figures/pictures/img_2790_pol.jpg}
    \end{subfigure}
    \caption{Docking area.}
\end{figure}
\vspace{-.5cm}

\begin{figure}[H]
    \begin{subfigure}[T]{.49\textwidth}
        \includegraphics[width=\textwidth]{figures/pictures/img_7458_s0.jpg}
    \end{subfigure} \hfill
    \begin{subfigure}[T]{.49\textwidth}
        \includegraphics[width=\textwidth]{figures/pictures/img_7458_pol.jpg}
        
    \end{subfigure}
    \caption{Wooden posts in the water.}
\end{figure}
\vspace{-.5cm}

\begin{figure}[H]
    \begin{subfigure}[T]{.49\textwidth}
        \includegraphics[width=\textwidth]{figures/pictures/img_11640_s0.jpg}
    \end{subfigure} \hfill
    \begin{subfigure}[T]{.49\textwidth}
        \includegraphics[width=\textwidth]{figures/pictures/img_11640_pol.jpg}
    \end{subfigure}
    \caption{Dock viewed from the water.}
\end{figure}
\vspace{-.5cm}

\begin{figure}[H]
    \begin{subfigure}[T]{.49\textwidth}
        \includegraphics[width=\textwidth]{figures/pictures/img_10170_s0.jpg}
    \end{subfigure} \hfill
    \begin{subfigure}[T]{.49\textwidth}
        \includegraphics[width=\textwidth]{figures/pictures/img_10170_pol.jpg}
    \end{subfigure}
    \caption{Underexposed image of a dock seen from the water.}
\end{figure}
\vspace{-.5cm}


\begin{figure}[H]
    \begin{subfigure}[T]{.49\textwidth}
        \includegraphics[width=\textwidth]{figures/pictures/img_3726_s0.jpg}
    \end{subfigure} \hfill
    \begin{subfigure}[T]{.49\textwidth}
        \includegraphics[width=\textwidth]{figures/pictures/img_3726_pol.jpg}
    \end{subfigure}
    \caption{Pollen on the water surface.}
\end{figure}
\vspace{-.5cm}

\begin{figure}[H]
    \begin{subfigure}[T]{.49\textwidth}
        \includegraphics[width=\textwidth]{figures/pictures/img_4038_s0.jpg}
    \end{subfigure} \hfill
    \begin{subfigure}[T]{.49\textwidth}
        \includegraphics[width=\textwidth]{figures/pictures/img_4038_pol.jpg}
    \end{subfigure}
    \caption{A docked boat seen from the water.}
\end{figure}
\vspace{-.5cm}

% \begin{figure}[H]
%     \begin{subfigure}[T]{.49\textwidth}
%         \includegraphics[width=\textwidth]{figures/pictures/img_5742_s0.jpg}
%     \end{subfigure} \hfill
%     \begin{subfigure}[T]{.49\textwidth}
%         \includegraphics[width=\textwidth]{figures/pictures/img_5742_pol.jpg}
%     \end{subfigure}
%     \caption{Building with glass windows.}
% \end{figure}
% \vspace{-.5cm}
\begin{figure}[H]
    \begin{subfigure}[T]{.49\textwidth}
        \includegraphics[width=\textwidth]{figures/pictures/img_1116_s0.jpg}
    \end{subfigure} \hfill
    \begin{subfigure}[T]{.49\textwidth}
        \includegraphics[width=\textwidth]{figures/pictures/img_1116_pol.jpg}
    \end{subfigure}
    \caption{A car standing next to a puddle of water.}
\end{figure}
\vspace{-.5cm}


\begin{figure}[H]
    \begin{subfigure}[T]{.49\textwidth}
        \includegraphics[width=\textwidth]{figures/pictures/img_9222_s0.jpg}
    \end{subfigure} \hfill
    \begin{subfigure}[T]{.49\textwidth}
        \includegraphics[width=\textwidth]{figures/pictures/img_9222_pol.jpg}
    \end{subfigure}
    \caption{Stone floor with reflections.}
\end{figure}
\vspace{-.5cm}

% \begin{figure}[H]
%     \begin{subfigure}[T]{.49\textwidth}
%         \includegraphics[width=\textwidth]{figures/pictures/img_9306_s0.jpg}
%     \end{subfigure} \hfill
%     \begin{subfigure}[T]{.49\textwidth}
%         \includegraphics[width=\textwidth]{figures/pictures/img_9306_pol.jpg}
%     \end{subfigure}
%     \caption{High contrast inside environment.}
% \end{figure}
% \vspace{-.5cm}


\begin{figure}[H]
    \begin{subfigure}[T]{.49\textwidth}
        \includegraphics[width=\textwidth]{figures/pictures/img_4722_s0.jpg}
    \end{subfigure} \hfill
    \begin{subfigure}[T]{.49\textwidth}
        \includegraphics[width=\textwidth]{figures/pictures/img_4722_unpol.jpg}
    \end{subfigure}
    \caption{Image where the linarly polarized light is removed to see whats under the water surface.}
\end{figure}
\vspace{-.5cm}

