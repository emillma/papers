\documentclass[british,titlepage,twoside]{ntnuthesis}

\title{Paving the Way for Enhanced Situational Awareness in the Maritime Domain: 
Designing and Ensembling a Sensor Rig for Multi-Sensor Data Set Acquisition}
\shorttitle{Sensor Rig for Multi-Sensor Data Set Acquisition}
\author{Emil Martens}
\shortauthor{Emil Martens}
\date{Spring 2023}
\addbibresource{thesis.bib}

\input{commands.tex}
\makeglossaries

\setabbreviationstyle[acronym]{long-postshort-user}
\glssetcategoryattribute{acronym}{nohyperfirst}{true}
\setabbreviationstyle{short-nolong}


% --------------------
% ---- Glossaries ----
% --------------------
\newglossaryentry{asyncio}{name=Asyncio, description={A Python library for asynchronous code.}}

% --------------------
% ----- Acronyms -----
% --------------------
\newacronym{asv}{ASV}{Autonomous Surface Vehicle}

\glsaddall
\makeglossaries

% \glsunset{cpu}
% \glsunset{gpu}
% \glsunset{lla}

% --------------------
% ----- Shortcuts ----
% --------------------

 

% \oneside
% 
\emergencystretch=1em
% \includeonly{chapters/10_intro/__include__}
% \includeonly{chapters/30_debayer_in_cuda/__include__}

\begin{document}
\includepdf{chapters/10_intro/frontpage.pdf}
\tableofcontents
% \listoffigures
% \listoftables
% \lstlistoflistings 
\pagebreak

\chapter{Efficient Preprocessing of Raw Image Data in CUDA}
\label{chap:debayer}

To compress the video frames from the cameras, it is necessary to first transform the data into a format that is compatible with the encoder.
This involves debayering the data to extract color information, converting the extracted color into a different color space, and packaging the data in a specific format.
Since the cameras utilize a specialized image sensor, a custom preprocessor was developed to perform these operations.
The preprocessor takes the raw bytes from the cameras as input and generates bytes that can be directly sent to the H.265 encoder.
In order to achieve the required throughput and limit power consumption, the preprocessor is implemented using\gls{cuda}.

This chapter provides an overview of the theory behind debayering and color spaces and presents the optimized\gls{cuda} implementation.


\section{Extracting color from raw bayer images}
\subsection{Bayer patterns}
To capture color images conventional cameras use \glspl{cfa}.
The most common type of \gls{cfa} is the Bayer pattern where green pixels cover half the array in a lattice, and the red and blue pixel locations are spaced between the green pixels as shown in Figure \ref{fig:bayer_pattern} \cite{getreuerMalvarHeCutlerLinearImage2011}.
Different orderings of the colors in the \gls{cfa} exists, with the most common ones being RGGB, BGGR, GRBG, and GBRG where the letters designate the order of the four top left pixeld in the image.

\begin{figure}[H]
    \centering
    \begin{tabular}[b]{ccc}
        \subcaptionbox{RGGB bayer pattern. \label{fig:bayer_pattern}}{\includegraphics[width=0.25\textwidth]{figures/debayer/bayer_pattern.pdf}
        }                                                                                                                                  &
        \subcaptionbox{Green at red \label{fig:mhc_gr}}{\includegraphics[width=0.25\linewidth]{figures/debayer/g_at_r.png}}                &
        \subcaptionbox{Green at blue \label{fig:mhc_gb}}{\includegraphics[width=0.25\textwidth]{figures/debayer/g_at_b.png}}                 \\
        \subcaptionbox{Red at green, red row \label{fig:mhc_rgr}}{\includegraphics[width=0.25\textwidth]{figures/debayer/r_at_g_rr.png}}   &
        \subcaptionbox{Red at green, blue row \label{fig:mhc_rgb}}{\includegraphics[width=0.25\textwidth]{figures/debayer/r_at_g_br.png}}  &
        \subcaptionbox{Red at blue \label{fig:mhc_rb}}{\includegraphics[width=0.25\textwidth]{figures/debayer/r_at_b.png}}                   \\
        \subcaptionbox{Blue at green, red row \label{fig:mhc_bgr}}{\includegraphics[width=0.25\textwidth]{figures/debayer/b_at_g_rr.png}}  &
        \subcaptionbox{Blue at green, blue row \label{fig:mhc_bgb}}{\includegraphics[width=0.25\textwidth]{figures/debayer/b_at_g_br.png}} &
        \subcaptionbox{Blue at red \label{fig:mhc_br}}{\includegraphics[width=0.25\textwidth]{figures/debayer/b_at_r.png}}
    \end{tabular}
    \caption{Bayer pattern and coefficient values used by Malvar-He-Cutler scaled by 8 \cite[Figure 3]{getreuerMalvarHeCutlerLinearImage2011}. The last row in the source is incorrect, and has been corrected here.}
    \label{fig:debayer:malvar_filters}
\end{figure}

\subsection{Image demosaicing}
Image demosaicing is the process of estimating full-resolution color information for an image that has been captured with a bayer pattern \cite{liImageDemosaicingSystematic2008}, e.g. at each red pixel in the \gls{cfa} the green and blue intensities needs to be estimated.
Simple methods, such nearest neighbours or linear interpolation, are prone to yielding images false colors and the checkboard like patterns called the zipper effect as shown in Figure \ref{fig:artifacts_gioia} \cite{gioiaDataDrivenConvolutionalModel2021} \cite{liImageDemosaicingSystematic2008}.
A multitude of more advanced methods have been proposed with increasing sofisication and performance \cite{liImageDemosaicingSystematic2008}.
Lately different deep learning methods has also proven to perform very well at this task \cite{kwanComparisonDeepLearning2019}.

\begin{figure}[H]
    \centering
    \includegraphics[width=0.5\textwidth]{figures/debayer/artifacts_gioia.png}
    \caption{Zipper effect and false colors \cite{gioiaDataDrivenConvolutionalModel2021}}
    \label{fig:artifacts_gioia}
\end{figure}

\subsection{Malvar-He-Cutler}
\gls{mhc} pattented a simple but performant linear method using 5x5 filters that shows surprisingly good results \cite{malvarHighqualityGradientcorrectedLinear2009}.
The method they present is derived as a modification of bilinear interpolation, and it involves adding Laplacian cross-channel corrections to improve the quality of the bilinear method \cite{getreuerMalvarHeCutlerLinearImage2011}.
The demosaicking is implemented by convolution with a set of linear filters, and there are eight different filters for interpolating the different color components at different locations, as shown in Figure \ref{fig:debayer:malvar_filters}.
Although several more sophisticated methods have been developed, the simple \gls{mhc} method works remarkably well \cite{liImageDemosaicingSystematic2008}\cite{kwanComparisonDeepLearning2019}\cite{getreuerMalvarHeCutlerLinearImage2011}.


\begin{figure}[H]
    \centering
    \begin{tabular}[b]{ccc}
        \subcaptionbox{Exact image}{\includegraphics[width=0.3\textwidth]{figures/debayer/house_orig.jpg}}                     &
        \subcaptionbox{Observed Image}{\includegraphics[width=0.3\textwidth]{figures/debayer/house_bayer.png}}                   \\
        \subcaptionbox{Bilinear (PSNR=25.61)}{\includegraphics[width=0.3\textwidth]{figures/debayer/house_bilinear.png}}       &
        \subcaptionbox{Hamilton-Adams (PSNR=31.62)}{\includegraphics[width=0.3\textwidth]{figures/debayer/house_hamilton.png}} &
        \subcaptionbox{Malvar-He-Cutler (PSNR=31.15)}{\includegraphics[width=0.3\textwidth]{figures/debayer/house_malvar.png}}
    \end{tabular}
    \caption{Coefficient values used by Malvar-He-Cutler scaled by 8 \cite{getreuerMalvarHeCutlerLinearImage2011}}
\end{figure}

\subsection{Shortcomings of MHC}
A shortcoming of the current method is that it does not use optimal gains.
Firstly the values used in \gls{mhc} are rounded to the neared dyadic rationals to work efficiently using integer arithmetic and bit-shifting \cite{getreuerMalvarHeCutlerLinearImage2011}.
This rounding is unnecessary as the currently proposed implementation uses floating point arithmetic.
Another minor issue is that the values in \gls{mhc} were found to be the best fit for the well-known public-domain Kodak image set \cite{malvarHighqualityGradientcorrectedLinear2009}.
This set of 24 varied images, shown in Figure \ref{fig:kodak_image_suite}, might not be the best representation of the images the \sr will capture in maritime environments.
It might be beneficial to recalculate the values based on a more representative dataset and use the floating point values rather than the rounded ones in the future.


\begin{figure}[H]
    \centering
    \includegraphics[width=0.8\textwidth]{figures/debayer/kodak_test_suite.png}
    \caption{Kodak image suite \cite{franzenTrueColorKodak2013}\cite{chungAdaptiveColorFilter2006}}
    \label{fig:kodak_image_suite}
\end{figure}


\section{Digital image representations}
As with everything else that is stored on a computer, images are encoded as a sequence of bits.
RGB images are popular representations, where each pixel is stored as three sequential bytes representing each color.
This representation is very intuitive and easy to understand, but other representations might be more suitable for certain applications.

\section{Luminance and chrominance}
In many use cases, each pixel's luminance (brightness) contains more helpful information than its chrominance (color), and separating these two components is beneficial.
This is relevant for visualization, as the human eye is more sensitive to changes in luminance than chrominance  \cite{lambWhyRodsCones2016}.
It can also apply to computer vision applications, where Lucas-Kanade is an example of an optical flow algorithm that only uses the luminance component \cite{lucasIterativeImageRegistration1981}.

YCbCr
\footnote{In the computer industry, the term YUV is widely used to refer to colorspaces that are encoded using YCbCr.}
is a popular color format where the Y component represents the luminance of the pixel, while the Cb and Cr components represent the blue-difference and red-difference chrominance information respectively, as depicted in Figure \ref{fig:ycbcr_example}.

\begin{figure}[H]
    \centering
    \includegraphics[width=\textwidth]{figures/debayer/YCbCr_example.pdf}
    \caption{Visualization of channels in a YCbCr image \cite{photoEnglishJohnMoulton2004}.}
    \label{fig:ycbcr_example}
\end{figure}

\subsubsection{Chroma subsampling}
Chroma subsampling is a technique that involves sampling the chrominance components (Cb and Cr) at a lower resolution than the luminance component (Y), as depicted in Figure \ref{fig:chroma_subsampling}.
A commonly used subsampling scheme is 4:2:0, where the chrominance components are sampled at half the resolution of the luminance component in both the horizontal and vertical directions \cite{ChromaSubsampling2023}.

During the process of debayering, redundant information is generated as each pixel in the output image contains three color channels, compared to only one in the raw Bayer image.
With this redundancy it is reasonable to perform chroma subsampling as the luminance component is sampled at every pixel, while a group of four pixels is required to obtain color information

\begin{figure}[H]
    \centering
    \includegraphics[width=.6\textwidth]{figures/debayer/chroma_subsampling.pdf}
    \caption{Chroma subsampling \cite{stevo-88EnglishMostWidely2010}.}
    \label{fig:chroma_subsampling}
\end{figure}


\subsection{Analog legacy}
It is important to note that certain color formats have their origins in the analog signal era.
One notable example is the BT.601 color format, a variant of YCbCr.
In this format, the Y component is represented by values ranging from 16 to 235, while the Cb and Cr components range from 16 to 240 \cite{YCbCr2023}.
The additional headroom and foot room within the byte is specifically allocated to accommodate transient signals, such as filter overshoots, and prevent undesirable effects like clipping \cite{Rec6012023}.
By reserving these values, the color representation remains within acceptable limits even when unexpected analog signal fluctuations occur.


\section{Interleved, Planar and Semi-Planar Packaging}
There are three main ways to store the color channels in memory, interleaved, planar and semi-planar \cite{baranYUVFormats2018}.
In interleaved formats, the color channels are stored in sequence, i.e.
$R_1 G_1 B_1 R_2 G_2 B_2 R_3 G_3 B_3$.
In planar formats, the color channels are stored in separate arrays, i.e.
$R_1 R_2 R_3 B_1 B_3 B_2 G_1 G_2 G_3$.
Semi-planar formats are a mix of the two where some channels are interleaved, and some are planar, i.e.
$R_1 R_2 R_3 G_1 B_1 G_2 B_2 G_3 B_3$.
The reson one might be preferred over the other is to optimize for memory locality.
For instance, if you are performing per-bit operations, having the bits of the same color channel next to each other is beneficial.

\begin{figure}[H]
    \centering
    \includegraphics[width=.8\textwidth]{figures/debayer/YUV_packaging.png}
    \caption{Visualization of I420 and NV12, two popular YUV 4:2:0 formats with planar and semi-planar packaging respectively \cite{baranYUVFormats2018}.}
    \label{fig:image_packaging}
\end{figure}

\subsection{Bit depth and packaging}
A final important property of image formats is the bit depth, i.e.
how many bits are used to represent each pixel, and how the bits are stored.
Most images use a single byte (8 bits) for each color channel, but higher bit depths offer better color fidelity and dynamic range.
If the number of bits is not divisible by 8, like for 10-bit images, the most space-efficient way to store and send them is to pack them into 8-bit bytes, i.e.
four 10-bit values are stored in 5 bytes.
However, as most \glspl{alu} does not support 10-bit operations, padding to the nearest full byte is better for computations.
% \subsection{Image data formats}
% As everything else, images are encoded as a sequence of bits.
% How we structure these bits in memory is another important property of image formats and can be very important for performance.
% Normal RGB images are encoded as three 8-bit values for each pixel, one for each color channel.


% \subsubsection{Interleved vs planar vs semi-planar}
% There are three main ways to store the color channels in memory, interleved, planar and semi-planar \cite{baranYUVFormats2018}.
% In interleved formats, the color channels are stored in sequence, i.e. RGBRGBRGB.
% In planar formats, the color channels are stored in separate arrays, i.e. RRRBBBGGG.
% Semi-planar formats are a mix of the two where some channels are interleved and some are planar, i.e. RRRGBGBGB.
% The reson one might be preferred over the other is to optimize for memory locality.
% For instance, if you are performing per-bit operations, having the bits of the same color channel next to each other is beneficial.


% \begin{figure}[H]
%     \centering
%     \includegraphics[width=.8\textwidth]{figures/debayer/YUV_packaging.png}
%     \caption{Visualization of I420 and NV12, two popular YUV 4:2:0 formats \cite{baranYUVFormats2018}.}
%     \label{fig:image_packaging}
% \end{figure}

% \subsubsection{Bit depth and packaging}
% A final importnt property of image formats is the bit depth, i.e. how many bits are used to represent each pixel.
% Most images use a single byte (8 bits) for each color channel, but higher bit depths offer better color fidelity and dynamic range.
% How these bits are packaged might also vary.
% If the number of bits is not divisable by 8, like for 10-bit images, the most space efficient way to store and send them is to pack them into 8-bit bytes, i.e. 4 10-bit values are stored in 5 bytes.
% However as most \glspl{alu} do not support 10-bit operations, padding the to the neares full byte is better for computations.


\section{Implementation Outline}
The goal is to translate the raw data from the cameras into an image format that is compatible with the \gls{h265} encoder, specifically the \gls{p010} format.
The \gls{p010} format is a YCbCr 4:2:0 format with interlaced chroma data.
In this format, each channel is represented by 10 bits per pixel stored in the most significant bits of 16-bit little-endian unsigned integers.

This conversion can be divided into five parts: unpacking the raw bits from the camera, separating different polarization angles, performing debayering, converting to YCbCr color space, applying chroma subsampling, and finally, arranging the data in the correct format.
The entire process is visualized in Figure \ref{fig:transform}.
This approach is inspired by the work of Vy Nguyen et al., where they perfom the color demosaicing on the polarization channels separately \cite{nguyenTwoStepColorPolarizationDemosaicking2022}.


\begin{figure}[H]
    \centering
    \includegraphics[width=\textwidth]{figures/polarized_image/transform.pdf}
    \caption{Visualization of how raw data is transformed into a YCbCr format where the angles of polarization are stacked vertically.}
    \label{fig:transform}
\end{figure}


Initially, I attempted to perform these steps using \gls{numpy} for data unpacking and \gls{opencv} for the transformations.
However, even when utilizing all eight \gls{cpu} cores on the \jx to their full extent, I was unable to achieve the required performance.
Furthermore, running the \gls{cpu} cores at maximum frequency had a substantial negative impact on the power consumption of the \jx.

To attain the necessary throughput, I made the decision to implement the whole translation process using\gls{cuda}.
This implementation is divided into two parts: data unpacking and transformation.
While\gls{cuda}-accelerated \gls{opencv} was considered, significant optimization potential was identified due to the unique format of the camera sensors.

\section {Unpacking Data from the Camera}
\subsection{Bit Pattern}
\label{sec:unpacking}
The first step in the transformation process is to unpack the raw data from the camera.
The \lucid cameras feature a 12-bit \gls{adc} and offer 23 different output formats with varying bit depths and packing.
As the \gls{h265} encoder supports 10-bit data, the \code{Mono10p} output format was chosen for the cameras \cite[17 ]{nvidiaNVIDIAJetsonAGX2019}.
This format densely packs the 10-bit data, as depicted in Figure \ref{fig:mono10p}, maximizing the network throughput.

\begin{figure}[H]
    \centering
    \subcaptionbox{Pixel data.}{\includegraphics[width=\textwidth]{figures/unpacking/layout_10p.pdf}}
    \subcaptionbox{Bytes sent over ethernet.}{\includegraphics[width=\textwidth]{figures/unpacking/layout_10p_sent.pdf}}
    \caption{Bit layout of the \code{Mono10p} format.}
    \label{fig:mono10p}
\end{figure}

I encountered difficulties in locating documentation regarding the bit ordering on Lucid's website.
As a workaround, I relied on two test images provided by the \cam.
These test images contained pixel values that increased monotonically, as depicted in Figure \ref{fig:test_pattern}.
By analyzing the data in the first line in these images, I was able to deduce the bit ordering.
Regrettably, I later discovered that Lucid Vision does provide separate documentation on pixel formats; however, it did not appear in their own search engine for unknown reasons \cite{lucidvisionlabsPixelFormatsLUCID2020}.

\begin{figure}
    \centering
    \includegraphics[width=0.4\textwidth]{figures/unpacking/test_pattern0.jpg}
    \includegraphics[width=0.4\textwidth]{figures/unpacking/test_pattern2.jpg}
    \caption{Two test images used to infer the bit ordering.
        The \cam can output several different test patterns useful for various testing purposes \cite{lucidvisionlabsTritonMPPolarized2020}.}
    \label{fig:test_pattern}
\end{figure}


\subsection{Bit Unpacking}
The pattern in Figure \ref{fig:mono10p} was identified as representing a little endian unsigned integer.
By reordering the bytes visually the bit pattern in Figure \ref{fig:mono10p} becomes more intuitive, as shown in Figure \ref{fig:mono10p_reordered}.

\begin{figure}
    \centering
    \includegraphics[width=\textwidth]{figures/unpacking/layout_10p_be.pdf}
    \caption{More intuitive visualization of the bit ordering in Figure \ref{fig:test_pattern}.}
    \label{fig:mono10p_reordered}
\end{figure}

As the\gls{cuda} GPU architecture uses little-endian representation, this is beneficial as we can interpret the incoming bits as a sequence of word-sized (32-bit) unsigned integers directly \cite[127]{nvidiaCUDAProgrammingGuide}.
The first three pixel values are then read as from the 30 least significant bits of the first word, the fourth pixel value is read from the most significant bits of the first word and the least eight significant bits of the second word, and so on.
This can be extracted using bit manipulation as shown in Listing \ref{lst:unpacking_half2}.
To get proper byte alignment, each thread unpacks five words (160 bits) corresponding to 16 pixel values.

\begin{listing}
    \begin{minted}{cuda}
        word_a = data[idx]; //copy from global memory
        word_b = data[idx+1]; //copy from global memory
        data[0] = __halves2half2(word_a & 0b1111111111,
                                 word_a >> 10 & 0b1111111111);
        data[1] = __halves2half2(word_a >> 20 & 0b1111111111;
                                 word_a >> 30 & 0b11 | (word_b & 0b11111111) << 2);
        data[2] = __halves2half2(word_b >> 8 & 0b1111111111,
                                 word_b >> 18 & 0b1111111111);
    \end{minted}
    \caption{How the first six pixel values are unpacked and cast to \gls{half2}. Section \ref{sec:half2} explains the use of \gls{half2} data type.}
    \label{lst:unpacking_half2}
\end{listing}

\section{From Bayer to YCbCr 4:2:0}
\label{sec:bayertoycbcr}
Three steps are needed to transform the Bayer data to YCbCr 4:2:0.
First, the Bayer data is debayered using the \gls{mhc} coefficients in Figure \ref{fig:debayer:malvar_filters}, then the color space is converted to YCbCr, and finally, chroma subsampling is applied.

An algebraic formulation using \gls{sympy} was developed to optimize the performance of this sequential transformation.
It was discovered that better performance could be achieved by calculating combined coefficients ahead of compilation.
This is because all three steps (debayering, color space conversion, and subsampling) consist exclusively of linear equations.
Table \ref{table:debayer_coefficients} provides a comparison of the number of \gls{fma} operations required when the steps are performed separately versus when they are combined.
Combining the operations also removed the need for synchronization, as each value is calculated independently.


\begin{table}
    \begin{minipage}[b]{.5\linewidth}
        \subcaptionbox{Separate debayering and color space conversion.
            Also shows where synchronization would be needed in\gls{cuda}.}{
            \footnotesize
            \begin{tabular}{|l|c| c|}
                \hline
                \textbf{Operation}                         & \textbf{\# of \glsxtrshort{fma}} \hspace{.4cm} \\
                \hline
                Green at red            (\ref{fig:mhc_gr}) & 9                                              \\
                Green at blue           (\ref{fig:mhc_gb}) & 9                                              \\
                Red at green 1 (\ref{fig:mhc_rgr})         & 11                                             \\
                Red at green 2 (\ref{fig:mhc_rgb})         & 11                                             \\
                Red at blue             (\ref{fig:mhc_rb}) & 9                                              \\
                Blue at green 1 (\ref{fig:mhc_bgr})        & 11                                             \\
                Blue at green 2 (\ref{fig:mhc_bgb})        & 11                                             \\
                Blue at red             (\ref{fig:mhc_br}) & 9                                              \\
                \textbf{\textit{Synchronize}}              &                                                \\
                Convert to YCbCr                           & 4$\times$9=36                                  \\
                \textbf{\textit{Synchronize}}              &                                                \\
                Average Cb                                 & 4                                              \\
                Average Cr                                 & 4                                              \\
                \hline
                \textbf{Total}                             & 124                                            \\
                \hline
            \end{tabular}}
    \end{minipage}
    \begin{minipage}[b]{.5\linewidth}
        \subcaptionbox{Joined debayering and color space conversion. No synchronization needed.}{
            \footnotesize
            \begin{tabular}{|l|c|}
                \hline
                \textbf{Operation} & \textbf{\# of \glsxtrshort{fma}} \hspace{.4cm} \\
                \hline
                Y at red           & 13                                             \\
                Y at green 1       & 13                                             \\
                Y at green 2       & 13                                             \\
                Y at blue          & 13                                             \\
                Averaged Cb        & 24                                             \\
                Averaged Cr        & 24                                             \\
                \hline
                \textbf{Total}     & 100                                            \\
                \hline
            \end{tabular}}
    \end{minipage}
    \caption{Comparison of the number of \gls{fma} operations required to get the desired output.}
    \label{table:debayer_coefficients}
\end{table}

\subsection{Automatic Code Generation in Python using Sympy}
\label{sec:code_generation}
With the algebraic formulation available, it was possible to write a custom code generator that would generate the\gls{cuda} code needed for the six functions needed to calculate the 6 YCbCr 4:2:0 values for a single \gls{cg}.
Listing \ref{listing:generated_function} shows parts of the generated code for calculating the Cr value for a \gls{cg}, and Figure \ref{fig:transformation} shows the pixel values that are accessed to calculate the values for one \gls{cg}.
A big advantage of algebraic formulation and automatic code generation is that it is easy to change what's calculated.
When it was discovered that the selected color profile was wrong in Section \ref{sec:inspection_and_comparison_of_the_final_output}, the only thing that needed to be changed to generate the correct code was the color space conversion matrix.

\begin{listing}[H]
    \begin{minted}{cuda}
            __device__ __forceinline__ __half2 get_Cr(__half2 **data, int col) {
                __half2 tmp = __float2half2_rn(5.000000000e-1f);
                tmp = __hfma2(__float2half2_rn(9.466415405e-5f), data[1][col + 1], tmp);
                tmp = __hfma2(__float2half2_rn(9.466415405e-5f), data[3][col - 1], tmp);
            \end{minted}
    \vspace{-28pt}
    \begin{minted}[linenos=false, autogobble=false]{cuda}
                ...
            \end{minted}
    \vspace{-28pt}
    \begin{minted}[firstnumber=24]{cuda}
        tmp = __hfma2(__float2half2_rn(-1.122532265e-5f), data[0][col], tmp);
        tmp = __hfma2_sat(__float2half2_rn(-1.122532265e-5f), data[2][col - 2], tmp);
        return __hfma2(__float2half2_rn(1023.0f), tmp, __float2half2_rn(0.0f));
        }
    \end{minted}
    \caption{Generated transformation function used to calculate Cr value for a \gls{cg}.
        Section \ref{sec:half2} explains the use of the \gls{half2} data type.}
    \label{listing:generated_function}
\end{listing}
\begin{figure}[H]
    \centering
    \includegraphics[width=.35\textwidth]{figures/polarized_image/normal_conv.pdf}
    \caption{Visualization of which pixels are used (white circle) to calculate the 6 YCbCr 4:2:0 values for the center \gls{cg} (thick white circle).}
    \label{fig:transformation}
\end{figure}
\section{Optimizations}
\subsection{Efficient Separation}
The \gls{volta} has 48KiB of available shared memory per block \cite{rigerunNVIDIAJetsonXavier2023}.
With an image width of 2448 pixels \cite{lucidvisionlabsTriton0MPPolarization} it is only possible to store 10 lines in local shared memory as each pixel uses 16 bits.
\begin{equation*}
    \frac{48Kib}{2448px/line * 16b/px} = \frac{393216b}{39168b/line} \approx 10.04line
\end{equation*}

In order to allocate local memory for prefetching, it became necessary to reduce the amount of image data stored in local memory.
This challenge was overcome by recognizing that the transformation process for even and odd lines is independent and identical.
As illustrated in Figure \ref{fig:saperation}, the even and odd lines can be separated, allowing two thread blocks to process each part independently.

It is important to highlight that the updated \jo now supports a maximum of $163KiB$ of local memory per thread block, rendering this separation unnecessary if utilized \cite{CUDA2023}.
\begin{figure}[H]
    \centering
    \includegraphics[width=.8\textwidth]{figures/polarized_image/separation.pdf}
    \caption{How separation is done in order to reduce the amount of shared memory required per thread block.}
    \label{fig:saperation}
\end{figure}

\subsection{Array Rotation and Concurrent Prefetching}
The current \gls{mhc} method requires six lines of the image to be available in local memory for the convolution process.
The local memory is structured as an array of pointers to local line data, making it possible to rotate the array after each convolution, as shown in Figure \ref{fig:reuse}.

\begin{figure}[H]
    \centering
    \includegraphics[width=\textwidth]{figures/polarized_image/rolling.pdf}
    \caption{Visualization of image rotation.}
    \label{fig:reuse}
\end{figure}

The local memory has eight rows, such that while computation is performed on the first six rows, the next two rows can be prefetched from global memory and unpacked.
This is done by a separate group of warps, enabling concurrent computation and prefetching, as shown in Figure \ref{fig:saperation}.


\begin{figure}[H]
    \centering
    \subcaptionbox{Initial implementation.}{\includegraphics[width=\textwidth]{figures/debayer/debayer_non_concurrent.pdf}}
    \subcaptionbox{Current implementation.}{\includegraphics[width=\textwidth]{figures/debayer/debayer_concurrent.pdf}}
    \caption{Concurrent computation and prefetching.}
    \label{fig:saperation}
\end{figure}




\subsection{Half-precision floating-point numbers}
\label{sec:half2}
A significant performance improvement was achieved by transitioning from single-precision floating-point numbers to half-precision floating-point numbers.
Half-precision floating-point numbers are 16-bit floating-point numbers with 11 bits of significand precision, which is enough for the 10-bit pixel values.
The utilization of half-precision floating-point numbers yields notable gains due to the dedicated hardware present in the \gls{alu} on the \gls{volta} \gls{gpu} \cite{CUDA2023}.
\footnote{This should not be confused with Tensor cores, which also operate on half-precision floating-point numbers but are not employed in this project}.
This dedicated hardware allows for \gls{simd} operations on a specialized data format called \gls{half2} that stores two half-precision floats, almost doubling throughput \cite{nvidiaHalf2ArithmeticFunctions2023}\cite{hoExploitingHalfPrecision2017}.

Using \gls{half2} is beneficial in this context, as each row consists of alternating polarization channels.
This means that the incoming data can be cast directly to \gls{half2} during the unpacking phase, as follows:



\begin{figure}[H]
    \centering
    \includegraphics[width=0.45\textwidth]{figures/polarized_image/half2_conv.pdf}
    \caption{Vusialization of the convolution performed on the \gls{half2} data type. The pairs of pixels encircled are stored in a single \gls{half2} variable.}
    \label{fig:half2_conv}
\end{figure}

A disadvantage of using \gls{half2} is that every arithmetic operation, like multiplication, has to be performed using special function calls \cite{nvidiaHalf2ArithmeticFunctions2023}.
To solve this, the code generator, discussed in Section \ref{sec:code_generation}, was modified to generate the correct function calls as shown in Listing \ref{listing:generated_function}.

\subsubsection{Small Infinity Bug}
Half-precision floating points consider any value exceeding $65504$ as infinity \cite{HalfprecisionFloatingpointFormat2023}.
This can pose a problem when performing calculations that might exceed this threshold during intermediate steps, resulting in the value being treated as infinity.
In the preprocessing phase, this issue occasionally arises since the \code{P010_10LE} format utilizes the ten most significant bits of a 16-bit integer, approaching the limit of half-precision floating points.

This was particularly problematic when the resulting values were cast to integers, as the resulting value would be zero, causing unexpected black spots in the image.
To mitigate this problem, the floating-point values are now cast to integers within the range of 0 to 1023, followed by a left bitshift of 6 bits to conform to the \code{P010_10LE} format.



\section{Failed optimizations}
Two memory optimizations were attempted without achieving any performance gains.
They still provided useful information on how

\subsection{Contiguous acces using warp level primitives} \label{sec:contuguous_access}
As we want to operate on 32-bit values and each pixel is stored as a 10-bit value, each thread is processing 160 bits, or five words, as it is the lowest common multiple of 32 and 10.
Thus every thread reads five consecutive words as shown:
\begin{align}
    a_T[i] = d[T*5+i], &  & i \in (0,1,2,3,4)
\end{align}
Where $T$ is the thread index in the warp, $a_T$ is the local memory of thread $T$, and $d$ is the relevant segment of the image stored in device memory.

It was hypothesized that it would be faster to let first read the data contiguously into shared memory and then redistribute it as follows:
\begin{align}
    s[i*32+T] & = d[i*32+T],  & i & \in (0,1,2,3,4) \\
    a_T[i]    & = s[(T*5+i)], & i & \in (0,1,2,3,4)
    \label{eq:contiguous_reading}
\end{align}
Where $s$ is shared local memory.
However, this was not the case as the performance was actually worse than the non-contiguous reading.


A second attempt was done using the \code{__shfl_sync} function, which is a warp-level primitive used to exchange data between threads in a warp \cite{linUsingCUDAWarpLevel2018}.
As the data exchange is performed directly between registers, this is faster than going through shared memory \cite{linUsingCUDAWarpLevel2018}
Data was read contiguously into local buffers of each thread, then exchanged so every thread ended up with five consecutive words.
However, finding the right indices is hard as you need to specify what data to send and what thread to read from, rather than what data to read from what thread.
The full index table, shown in Table \ref{table:memory_index} in Appendix \ref{chap:additional_resources}, was created and studied in order to end up with the following formulation:
\begin{align}
    c_T[i] & = d[i*32+T],       &   &                   & i & \in (0,1,2,3,4) \\
    c_T[j] & \rightarrow a_x[i] & j & = (2 * (T -i))\%5 & i & \in (0,1,2,3,4) \\
    a_T[i] & \leftarrow c_j[x]  & j & = (T*5 + i)\%32   & i & \in (0,1,2,3,4)
    \label{eq:contiguous_reading_shfl}
\end{align}
where $x$ represents the thread the data is sent to.
The code equivalent to this is written as follows.
\begin{minted}[linenos=false]{cuda}
    a[i] = __shfl_sync(0xffffffff, c[2(*(T-i))%5], (T*5 + i)%32);
\end{minted}



The reason why coalesced memory access using the \code{__shfl_sync} operation is slower than non-coalesced memory access is believed to be due to the effectiveness of caching in mitigating the cost of non-coalesced access. This hypothesis finds support in Volkov's thesis, specifically in the section titled "Unstructured memory access," where it demonstrates that the cost of unstructured memory access is significantly reduced for smaller data transactions \cite[Sec 6.7]{volkovLatencyHiding2016}.

To verify this hypothesis, I intended to utilize the \gls{nsight} tool to create a memory chart, like the one in Figure \ref{fig:cache_hits}. However, it was not feasible as this feature is only available for NVIDIA \glspl{gpu} with compute capability 7.5 or greater while the \jx has to compute capability 7.2 \cite{crovellaUsingNsightCompute2019}\cite{CUDA2023}.


\begin{figure}[H]
    \centering
    \includegraphics[width=\textwidth]{figures/cuda/cache_hits.png}
    \caption{Illustrative example of memorychart generated with \gls{nsight}. \cite{nv-computeNsightComputeMemory2022}}
    \label{fig:cache_hits}
\end{figure}


\subsection{Use of constant memory}
During the development process, it was tested whether storing the constant values used in the debayer algorithm in constant memory would speed up the process.
Constant memory is a type of limited read-only memory available on NVIDIA \glsps{gpu} \cite[61]{nvidiaCUDABestPractices2023}.
NVIDIA \glspl{gpu} only have 64KiB of constant memory \cite[61]{nvidiaCUDABestPractices2023}.
Read instructions from constant memory are very efficient, and the best performance is achieved when all threads in a warp, as opposed to regular memory where this would result in inefficient collisions \cite[61]{nvidiaCUDABestPractices2023} \cite[13,14]{volkovLatencyHiding2016}.
This would be the case for the debayer algorithm as all threads in a warp read the same constant values.

All unique constant variables used in the algorithm were collected as a part of the automatic code generation and stored in a constant device array.
Unfortunately, this change had no visible impact on the performance.
A minimal test was later created that performed a very simple repeated multiply and added operation, where the coefficient and constants were either stored in constant memory or defined as literal variables in the function as shown in \code{mfa_1} and \code{mfa_2} in Listing \ref{listing:cuda_mem_tests}.
These minimal tests showed that using constant memory was actually marginally slower (0.08\% on average) than using literal values.

Further, it was tested whether storing the coefficient in constant variables would improve performance as there is no built-in \code{__half2} literal type, as shown in \code{mfa_3} in Listing \ref{listing:cuda_mem_tests}.
This gave marginally better results than \code{mfa_1} (0.04\% on average) but was ignored as it was not worth the effort to implement in the code generator.

\begin{listing}[H]
    \begin{minted}{cuda}
        __device__ __forceinline__ __half2 mfa_1(__half2 a) {
        return __hfma2(__float2half2_rn(0.098f), a, __float2half2_rn(3.14f));
        }
        __device__ __forceinline__ __half2 mfa_2(__half2 a) {
            return __hfma2(constant_mem[0], a, constant_mem[1]);
        }
        __device__ __forceinline__ __half2 mfa_3(__half2 a) {
            const __half2 b = __float2half2_rn(0.098f);
            const __half2 c = __float2half2_rn(3.14f);
            return __hfma2(b, a, c);
        }
    \end{minted}
    \caption{Small functions used to test different memory implementations.}
    \label{listing:cuda_mem_tests}
\end{listing}





\chapter{Conclusion}
In this \master I have successfully finished the development of a \sr capable of capturing synchronized stereo data from two polarization cameras.
Using \gls{cuda} and \gls{gstreamer} I have achieved on-the-fly hardware-accelerated video compression on the \jx, which enables the recording of longer data sets.
In addition to this, the \sr is now easy to use for anyone, thanks to the new web-based \gls{gui} and 3D printed ergonomic handles.

Completing the \sr has been a challenging but rewarding process where I have acquired many new skills.
Based on the experience I have gained, I can say that designing and building a \sr is a non-trivial and time-consuming task that should be avoided if existing datasets are sufficient for the task at hand.

The \sr provides a combination of sensor data that is believed to be unique and valuable for future research.
My supervisor, Annette Stahl, and I have started planning a scientific paper presenting this new type of data set and how stereo polarization cameras can be used to enhance situational awareness in the maritime domain.

The achievements of this \master are expected to be an excellent starting point for my future Ph.D. work and be used in several other projects at NTNU.




\newglossarystyle{mystyle}{\glossarystyle{long}\renewenvironment{theglossary}%
    {\footnotesize \begin{longtable}{p{3cm}p{\glsdescwidth}}}{\end{longtable}}%
}
\printglossary[style=mystyle,type=\acronymtype]
\printglossary[style=mystyle]
\AtNextBibliography{\footnotesize}
\printbibliography

\chapter{Additional resources}
\label{chap:additional_resources}
\section{Index Table used to Determine Memory Access Pattern}
The following table was used to determine the correct access pattern in Section \ref{sec:contuguous_access}.
\begin{table}[H]
    \small
    \begin{tabular}{r|rrrrr|rrrrr}
        T  & $c_T[0]$ & $c_T[1]$ & $c_T[2]$ & $c_T[3]$ & $c_T[4]$ & $a_T[0]$ & $a_T[1]$ & $a_T[2]$ & $a_T[3]$ & $a_T[4]$ \\
        \hline

        0  & $0$      & $32$     & $64$     & $96$     & $128$    & $0$      & $1$      & $2$      & $3$      & $4$      \\
        1  & $1$      & $33$     & $65$     & $97$     & $129$    & $5$      & $6$      & $7$      & $8$      & $9$      \\
        2  & $2$      & $34$     & $66$     & $98$     & $130$    & $10$     & $11$     & $12$     & $13$     & $14$     \\
        3  & $3$      & $35$     & $67$     & $99$     & $131$    & $15$     & $16$     & $17$     & $18$     & $19$     \\
        4  & $4$      & $36$     & $68$     & $100$    & $132$    & $20$     & $21$     & $22$     & $23$     & $24$     \\
        5  & $5$      & $37$     & $69$     & $101$    & $133$    & $25$     & $26$     & $27$     & $28$     & $29$     \\
        6  & $6$      & $38$     & $70$     & $102$    & $134$    & $30$     & $31$     & $32$     & $33$     & $34$     \\
        7  & $7$      & $39$     & $71$     & $103$    & $135$    & $35$     & $36$     & $37$     & $38$     & $39$     \\
        8  & $8$      & $40$     & $72$     & $104$    & $136$    & $40$     & $41$     & $42$     & $43$     & $44$     \\
        9  & $9$      & $41$     & $73$     & $105$    & $137$    & $45$     & $46$     & $47$     & $48$     & $49$     \\
        10 & $10$     & $42$     & $74$     & $106$    & $138$    & $50$     & $51$     & $52$     & $53$     & $54$     \\
        11 & $11$     & $43$     & $75$     & $107$    & $139$    & $55$     & $56$     & $57$     & $58$     & $59$     \\
        12 & $12$     & $44$     & $76$     & $108$    & $140$    & $60$     & $61$     & $62$     & $63$     & $64$     \\
        13 & $13$     & $45$     & $77$     & $109$    & $141$    & $65$     & $66$     & $67$     & $68$     & $69$     \\
        14 & $14$     & $46$     & $78$     & $110$    & $142$    & $70$     & $71$     & $72$     & $73$     & $74$     \\
        15 & $15$     & $47$     & $79$     & $111$    & $143$    & $75$     & $76$     & $77$     & $78$     & $79$     \\
        16 & $16$     & $48$     & $80$     & $112$    & $144$    & $80$     & $81$     & $82$     & $83$     & $84$     \\
        17 & $17$     & $49$     & $81$     & $113$    & $145$    & $85$     & $86$     & $87$     & $88$     & $89$     \\
        18 & $18$     & $50$     & $82$     & $114$    & $146$    & $90$     & $91$     & $92$     & $93$     & $94$     \\
        19 & $19$     & $51$     & $83$     & $115$    & $147$    & $95$     & $96$     & $97$     & $98$     & $99$     \\
        20 & $20$     & $52$     & $84$     & $116$    & $148$    & $100$    & $101$    & $102$    & $103$    & $104$    \\
        21 & $21$     & $53$     & $85$     & $117$    & $149$    & $105$    & $106$    & $107$    & $108$    & $109$    \\
        22 & $22$     & $54$     & $86$     & $118$    & $150$    & $110$    & $111$    & $112$    & $113$    & $114$    \\
        23 & $23$     & $55$     & $87$     & $119$    & $151$    & $115$    & $116$    & $117$    & $118$    & $119$    \\
        24 & $24$     & $56$     & $88$     & $120$    & $152$    & $120$    & $121$    & $122$    & $123$    & $124$    \\
        25 & $25$     & $57$     & $89$     & $121$    & $153$    & $125$    & $126$    & $127$    & $128$    & $129$    \\
        26 & $26$     & $58$     & $90$     & $122$    & $154$    & $130$    & $131$    & $132$    & $133$    & $134$    \\
        27 & $27$     & $59$     & $91$     & $123$    & $155$    & $135$    & $136$    & $137$    & $138$    & $139$    \\
        28 & $28$     & $60$     & $92$     & $124$    & $156$    & $140$    & $141$    & $142$    & $143$    & $144$    \\
        29 & $29$     & $61$     & $93$     & $125$    & $157$    & $145$    & $146$    & $147$    & $148$    & $149$    \\
        30 & $30$     & $62$     & $94$     & $126$    & $158$    & $150$    & $151$    & $152$    & $153$    & $154$    \\
        31 & $31$     & $63$     & $95$     & $127$    & $159$    & $155$    & $156$    & $157$    & $158$    & $159$
    \end{tabular}
    \caption{Relative location of image data stored in local memory. \newline e.g. \quad $a_1[1]=d[6+k]$ and  $c_1[1]=d[33+k]$.}
    \label{table:memory_index}
\end{table}
\section{Forum posts on PPS and compilation}\label{appendix:form_posts}

Following is a list of forum posts that were consulted during the compilation process.
The list is made by downloading and parsing the HTML from my activity log on the \gls{nforum} \cite{martensPostsRedEmil} and filtering out the relevant posts.
Out of the 544 posts visted, the 89 posts below matced one or more of the following regex search patterns:
\begin{itemize}
    \item \code{[^\w]pps}
    \item \code{boot}
    \item \code{flash}
    \item \code{kernel}
\end{itemize}
The total number of replies red is 1092.
Each post can be found by replacing the \code{<ID>} in the link below with the ID in the table.

\url{https://forums.developer.nvidia.com/t/<ID>}

\small
\begin{longtable}{p{.75\textwidth}rr}
    Title                                                                                               & Replies & ID     \\
    \hline                                                                                                                 \\
    AGX Xavier PPS fetch timeout R35.2.1(R32.7.1 works fine)                                            & 2       & 252374 \\
    About Flashing to a USB Drive                                                                       & 61      & 219316 \\
    Add PPS signal from gnss reciever to xavier nx                                                      & 11      & 155347 \\
    Add PPS signal to device tree in Nano                                                               & 5       & 169488 \\
    Boot Jetson AGX Xavier directly with configured SSD storage                                         & 11      & 219030 \\
    Boot from external drive                                                                            & 27      & 182883 \\
    Build the Real-Time Kernel                                                                          & 29      & 229571 \\
    CUDA 10.2 (cudart.so.10.2) missing (?) after FLASH install                                          & 9       & 186995 \\
    Can not flash Jetson xavier NX                                                                      & 6       & 183968 \\
    Can’t enable PPS in TX2 NX                                                                          & 8       & 196937 \\
    Can’t enable PPS on AGX Xavier                                                                      & 8       & 246834 \\
    Can’t enable PPS on Jetson AGX Xavier                                                               & 2       & 243842 \\
    Can’t find kernel\_src.tbz2                                                                         & 6       & 43288  \\
    Connecting GPS with PPS to Xavier                                                                   & 17      & 69356  \\
    Custom kernel build on Jetson AGX Xavier                                                            & 11      & 197023 \\
    Enable PPS on Jetson AGX Xavier                                                                     & 6       & 161342 \\
    Enable PPS on Jetson Linux 35.3.1 for Xavier AGX                                                    & 5       & 252416 \\
    Enabling PPS on Jetson Nano with Jetpack 4.3                                                        & 20      & 119418 \\
    Enabling PPS on Xavier AGX                                                                          & 31      & 147762 \\
    Error flashing SSD: /Linux\_for\_Tegra/bootloader/signed/flash.idx is not found                     & 3       & 240046 \\
    Error when flashing an NVMe ssd with initrd                                                         & 5       & 192031 \\
    Error: ECID read failed when using nvautoflash.sh                                                   & 3       & 212606 \\
    Facing problem while flashing the eMMC on the Jetson Nano production ready module (not the Dev Kit) & 28      & 80071  \\
    Failed to start Load Kernel Modules on Jetson Nano 4GB - L4T 32.7.2                                 & 8       & 223858 \\
    First Boot endlessly in “A start job is running for End-user configuration…”                        & 4       & 158015 \\
    Flash Issue - The target is in a bad state                                                          & 17      & 193746 \\
    Flash Jeston AGX Xavier with Command-Line Install with Docker on Windows 10 failed                  & 4       & 191870 \\
    Flash Jetpack 5.0.2 to SATA SSD                                                                     & 4       & 230802 \\
    Flash Jetson Agx Xavier Devkit to boot from SD Card                                                 & 9       & 196130 \\
    Flash jetpack 4.6 into sd card                                                                      & 4       & 194387 \\
    Flash kernel image on Nvidia TX2                                                                    & 7       & 60919  \\
    Flash of NVMe SSD                                                                                   & 7       & 189770 \\
    Flashing QSPI-NOR                                                                                   & 8       & 194180 \\
    Flashing Xavier with modified Device Tree                                                           & 11      & 80016  \\
    Flashing for booting from NVMe                                                                      & 7       & 210660 \\
    Flashing to NVMe SSD                                                                                & 9       & 210573 \\
    How to Boot from USB Drive?                                                                         & 50      & 172676 \\
    How to rebuild kernel for Jetson Linux 35.2.1                                                       & 5       & 243841 \\
    How to update cuda on TX2 without re-flashing TX2                                                   & 7       & 83298  \\
    How to use To use TEGRA\_AON\_GPIO in the kernel tegra x1??                                         & 5       & 68964  \\
    I try to flash nvme on my xavier nx, but failed and nothing on nvme                                 & 9       & 203664 \\
    Install kernel modules error"No rule to make target'modules\_install'. stop"                    & 6       & 214777 \\
    Is it possible to get a PPS output?                                                                 & 2       & 244787 \\
    JetPack 3.1 kernel source tag problem                                                               & 70      & 51929  \\
    JetPack 4.2 Flashing Issues and how to resolve                                                      & 67      & 73387  \\
    JetPack 5 and PPS on the Xavier NX                                                                  & 4       & 225824 \\
    JetPack 5.0.2,Orin AGX dev kit, /dev/pps1 not found                                                 & 11      & 230419 \\
    JetPack SDK in Docker for simple and clean flashing of a Jetson TX2                                 & 2       & 120526 \\
    Jetpack 4.2.1 fails to boot on Nano                                                                 & 9       & 78710  \\
    Jetpack 5.0.1 switch from Tegra 23x to 19x family SOC in kernel configuration                       & 2       & 219556 \\
    Jetpack 5.0.2 , boot from SATA SSD on Jetson Xavier NX. error: dev/sda1 not found                   & 27      & 238718 \\
    Jetpack-5.0.2 - Failed to flash NVMe with custom dtb and kernel’s Image                             & 8       & 238855 \\
    Jetpack5.0.2 kernel compile error                                                                   & 7       & 224160 \\
    Jetson Nano - Keeping time updated after reboot                                                     & 10      & 72380  \\
    Jetson TX2 failed to flash with unbuntu with SDK manager                                            & 3       & 185770 \\
    Jetson Xavier AGX 32GB flashing fails                                                               & 22      & 166931 \\
    Jetson Xavier AGX not booting after fstab modification                                              & 4       & 165516 \\
    Jetson Xavier Boot on SSD with Cboot                                                                & 20      & 73125  \\
    Jetson Xavier NX DEVKIT secureboot enabled                                                          & 21      & 158361 \\
    Jetson Xavier hardware pps\_out                                                                     & 18      & 124003 \\
    Jetson nano boot failed                                                                             & 7       & 73877  \\
    Jetson nano: PPS GPIO interrupt not getting registered                                              & 8       & 198085 \\
    Jetson tx2 cannot flash jetpack on wsl                                                              & 4       & 65739  \\
    Jetson xavier xn stuck in boot loop after trying to initrd flash to nvme SSD                        & 10      & 239392 \\
    Kernel build script nvbuild.sh with output dir option not working                                   & 3       & 173087 \\
    Move EMMC to NVMe boot: step 10 clarification                                                       & 5       & 188089 \\
    NTP with PPS Support on Xavier AGX                                                                  & 6       & 164887 \\
    New SDK Manager Flash Issue                                                                         & 6       & 81335  \\
    PPS kernel issue on Jetson Xavier AGX (L4T 34.1.1)                                                  & 5       & 224122 \\
    Please provide more detailed guidance to flash Jetson Nano with Windows 11 PC                       & 10      & 236808 \\
    Problem flashing AGX XAVIER                                                                         & 10      & 170912 \\
    R35.1 kernel + Tegra19x SOC family configuration compile errors                                     & 2       & 228026 \\
    Re-flash the device tree jetson agx xavier                                                          & 2       & 196565 \\
    Reboots while downloading the jetson-voice demo docker                                              & 5       & 125083 \\
    RuntimeError: CUDA error: no kernel image is available for execution on the device                  & 29      & 167708 \\
    SDK Manager flash to NVMe with flash checkpoint error                                               & 60      & 209375 \\
    Ssd boot on custom board with AGX SoM                                                               & 16      & 200732 \\
    System RAW image L4T 32.x.y l4t\_initrd\_flash.sh                                                   & 7       & 215356 \\
    TUTORIAL: Using sdkmanager for flashing on Windows via WSL2 + WSLg                                  & 2       & 225759 \\
    Unable to achieve PPS with Jetpack 5.0.2                                                            & 26      & 232101 \\
    Unable to flash Xavier AGX with SDK because of *** Reading ECID … *** Error: ECID read failed       & 2       & 189616 \\
    Unable to set breakpoints in kernels, debugging not working, VS2017 15.9.6                          & 5       & 70503  \\
    Using pps-gpio on Orin                                                                              & 3       & 225849 \\
    Xavier - Building the Kernel from Source                                                            & 2       & 77044  \\
    flash jetson tk1 using Windows                                                                      & 4       & 40331  \\
    how to setup NTPD with GPIO PPS                                                                     & 2       & 106942 \\
    keyword ‘return’ in cuda kernel what is its meaning?                                                & 5       & 19363  \\
    rootfsAB and test kernel/dtb                                                                        & 23      & 210108 \\
    “Cannot find matching device in recovery mode” error when flashing AGX Xavier                       & 21      & 219283
\end{longtable}



% \section{Compilation script}
\begin{minted}[fontsize=\tiny]{python}
    import subprocess
    from pathlib import Path
    import shutil
    import os
    from concurrent.futures import ThreadPoolExecutor, wait
    import requests
    
    password = input("Enter sudo password: ")
    
    
    def run(cmd: str, cwd=None):
        p = subprocess.run(cmd, cwd=cwd, env=env, shell=True, check=True)
        assert not p.returncode
    
    
    def run_sudo(cmd: str, cwd=None):
        p = subprocess.run(
            f"echo {password} | sudo -S {cmd}", cwd=cwd, env=env, shell=True, check=True
        )
        assert not p.returncode
        return p
    
    
    filedir = Path(__file__).parent
    dirs = [
        work_dir := filedir / "work",
        download_dir := filedir / "download",
        compiler_dir := work_dir / "l4t-gcc",
        l4t_dir := work_dir / "Linux_for_Tegra",
        public_dir := l4t_dir / "source/public",
        kernel_dir := l4t_dir / "source/public/kernel/kernel-5.10",
        rootfs_dir := l4t_dir / "rootfs",
        kernel_out := l4t_dir / "images",
        modules_out_dir := kernel_out / "modules",
        # sources_dir := l4t_dir / "sources",
    ]
    for d in dirs:
        d.mkdir(parents=True, exist_ok=True)
    
    
    cross_comp = compiler_dir / "bin/aarch64-buildroot-linux-gnu-"
    default_user_script = filedir / "l4t_create_default_user.sh"
    
    config = kernel_out / ".config"
    defconfig = kernel_dir / "arch/arm64/configs/defconfig"
    dtsi = (
        public_dir
        / "hardware/nvidia/soc/t19x/kernel-dts/tegra194-soc/tegra194-soc-base.dtsi"
    )
    pps_gpio_c = kernel_dir / "drivers/pps/clients/pps-gpio.c"
    
    env = os.environ.copy()
    env.update(
        {
            "CROSS_COMPILE": str(cross_comp),
            "LOCALVERSION": "-tegra",
            "TEGRA_KERNEL_OUT": str(kernel_out),
        }
    )
    
    
    def download(url: str, name: str):
        tmp_file = download_dir / f"tmp_{name}"
        if not (out_file := download_dir / name).exists():
            with requests.get(url, stream=True) as r, open(tmp_file, "wb") as f:
                shutil.copyfileobj(r.raw, f)
            shutil.move(tmp_file, out_file)
        return out_file
    
    
    common = "https://developer.nvidia.com/downloads/embedded/l4t/r35_release_v3.1"
    def get_bsp():
        url = f"{common}/release/jetson_linux_r35.3.1_aarch64.tbz2/"
        file = download(url, Path(url).name)
        run_sudo(f"tar -xjf {file} -C {work_dir}")
    
    def get_rootfs():
        url = f"{common}/release/tegra_linux_sample-root-filesystem_r35.3.1_aarch64.tbz2/"
        file = download(url, Path(url).name)
        run_sudo(f"tar -xjf {file} -C {rootfs_dir}")
    
    def get_bsp_sources():
        url = f"{common}/sources/public_sources.tbz2/"
        file = download(url, Path(url).name)
        run(f"tar -xjf {file}", work_dir)
        run("tar -xjf kernel_src.tbz2", l4t_dir / "source/public")
    
    def get_compiler():
        url = "https://developer.nvidia.com/embedded/jetson-linux/bootlin-toolchain-gcc-93"
        file = download(url, Path(url).name)
        run(f"tar -xf {file} -C {compiler_dir}")

    skipto = 0
    
    if skipto <= 0:
        futures = []
        with ThreadPoolExecutor() as executor:
            futures.append(executor.submit(get_bsp))
            futures.append(executor.submit(get_bsp_sources))
            futures.append(executor.submit(get_rootfs))
            futures.append(executor.submit(get_compiler))
            a, b = wait(futures)

    if skipto <= 1:
        for file in [pps_gpio_c, dtsi, defconfig]:
            shutil.copy(filedir / file.name, file)

    if skipto <=2:
        run("make mrproper", kernel_dir)
        kwargs = dict(
            ARCH="arm64",
            O=kernel_out,
            CROSS_COMPILE=cross_comp,
            KERNEL_OUT=kernel_out,
            LOCALVERSION="-tegra",
            INSTALL_MOD_PATH=str(kernel_out / "modules"),
        )
    
        common = f"make {' '.join(f'{k}={v}' for k, v in kwargs.items())} -j$(nproc)"
        run(f"{common} tegra_defconfig", kernel_dir)
        run(f"{common} Image", kernel_dir)
        run(f"{common} dtbs", kernel_dir)
        run(f"{common} modules", kernel_dir)
        run(f"{common} modules_install", kernel_dir)

    if skipto <=3:
        run_sudo(f"cp {kernel_out / 'arch/arm64/boot/Image'} {l4t_dir / 'kernel/'}")
        src = kernel_out / "arch/arm64/boot/dts/nvidia"
        dst = l4t_dir / "kernel/dtb"
        run_sudo(f"cp -a {src}/. {dst}")
    
    
    if skipto <=4:
        dst = l4t_dir / "kernel/kernel_supplements.tbz2"
        src = l4t_dir / "images/modules/lib/modules"
        dst.unlink(missing_ok=True)
        run_sudo(f"tar --owner root --group root -cjf {dst} {src}")
        run_sudo("/bin/bash apply_binaries.sh", l4t_dir)
        
    if skipto <=5:
        out = next((rootfs_dir).rglob("usr/lib/modules/*/kernel/drivers/gpu/nvgpu"))
        run_sudo(f"cp {kernel_out/'drivers/gpu/nvgpu/nvgpu.ko'} {out}")
        
        run_sudo(f"cp {default_user_script} {l4t_dir / default_user_script.name}")
        run_sudo(f"/bin/bash {l4t_dir / default_user_script.name} -p nvidia", l4t_dir)

    if skipto <=6:
        run_sudo("/bin/bash flash.sh jetson-agx-xavier-devkit mmcblk0p1", l4t_dir)

    if skipto <=7:
        file = l4t_dir / "tools/kernel_flash/l4t_initrd_flash.sh"
        layout = filedir / "flash_l4t_nvme.xml"
        size_gp = 520
        size_bytes = ((size_gp - 20) * 10**9 + 4096 - 1) // 4096 * 4096
        cmd = (
            f"/bin/bash {file} -c {layout}"
            " --external-device nvme0n1 --showlogs"
            " --no-flash"
            " --external-only"
            f" -S {size_bytes}"
            " jetson-agx-xavier-devkit nvme0n1p1"
        )
        run_sudo(cmd, l4t_dir)

    if skipto <=8:
        input("Put xavier into recovery mode and press enter")
        cmd = (
            f"/bin/bash {file} -c {layout}"
            " --external-device nvme0n1 --showlogs"
            " --flash-only"
            " --external-only"
            f" -S {size_bytes}"
            " jetson-agx-xavier-devkit nvme0n1p1"
        )
        run_sudo(cmd, l4t_dir)
\end{minted}
\section{Cura configurations}
\label{sec:cura_configurations}
Following is the complete set of parameters that were modified from their default values in Cura to achaive the desired print quality.
\begin{table}[H]
    \centering
    \scriptsize
    \begin{tabular}{ |l|r| }
        \hline
        \textbf{Parameter}               & \textbf{Value} \\
        \hline
        adhesion\_extruder\_nr           & 1              \\
        adhesion\_type                   & brim           \\
        bridge\_settings\_enabled        & False          \\
        jerk\_enabled                    & False          \\
        layer\_height                    & 0.16           \\
        layer\_height\_0                 & 0.3            \\
        material\_bed\_temperature       & 70             \\
        material\_shrinkage\_percentage  & 100            \\
        prime\_tower\_brim\_enable       & False          \\
        prime\_tower\_enable             & True           \\
        prime\_tower\_position\_x        & 220            \\
        prime\_tower\_size               & 20             \\
        support\_enable                  & True           \\
        support\_extruder\_nr            & 1              \\
        support\_interface\_extruder\_nr & 1              \\
        support\_type                    & buildplate     \\
        \hline
    \end{tabular}
    \caption{General configurations.}
\end{table}
\pagebreak
\begin{multicols}{2}
    \begin{table}[H]
        \scriptsize
        \begin{tabular}{ |l|r| }
            \hline
            \textbf{Parameter}                    & \textbf{Value} \\
            \hline
            acceleration\_print                   & 8000           \\
            acceleration\_roofing                 & 8000.0         \\
            acceleration\_topbottom               & 8000.0         \\
            acceleration\_travel                  & 8000.0         \\
            acceleration\_wall\_0                 & 8000.0         \\
            acceleration\_wall\_x                 & 8000.0         \\
            bottom\_layers                        & 5              \\
            bridge\_skin\_speed                   & 35.0           \\
            bridge\_wall\_speed                   & 35             \\
            brim\_gap                             & 0.4            \\
            brim\_line\_count                     & 10             \\
            cool\_fan\_speed                      & 60             \\
            cool\_min\_layer\_time                & 3              \\
            hole\_xy\_offset                      & 0.2            \\
            infill\_enable\_travel\_optimization  & True           \\
            infill\_pattern                       & trihex-        \\
                                                  & agon           \\
            infill\_sparse\_density               & 20             \\
            initial\_layer\_line\_width\_factor   & 120            \\
            jerk\_print                           & 30.0           \\
            material\_final\_print\_temperature   & 240.0          \\
            material\_flow\_layer\_0              & 75             \\
            material\_initial\_print\_temperature & 240.0          \\
            material\_standby\_temperature        & 240.0          \\
            prime\_tower\_min\_volume             & 5              \\
            retraction\_hop                       & 1.5            \\
            roofing\_layer\_count                 & 1              \\
            roofing\_monotonic                    & False          \\
            roofing\_pattern                      & lines          \\
            skin\_overlap                         & 10             \\
            speed\_infill                         & 70             \\
            speed\_layer\_0                       & 20             \\
            speed\_prime\_tower                   & 70.0           \\
            speed\_roofing                        & 70.0           \\
            speed\_topbottom                      & 70.0           \\
            speed\_wall\_0                        & 70.0           \\
            speed\_wall\_x                        & 70.0           \\
            support\_fan\_enable                  & False          \\
            support\_supported\_skin\_fan\_speed  & 60             \\
            support\_tree\_limit\_branch\_reach   & False          \\
            support\_xy\_distance\_overhang       & 0.4            \\
            switch\_extruder\_retraction\_amount  & 3              \\
            switch\_extruder\_retraction\_speed   & 40.0           \\
            switch\_extruder\_retraction\_speeds  & 40             \\
            top\_bottom\_pattern                  & lines          \\
            top\_bottom\_pattern\_0               & lines          \\
            top\_layers                           & 5              \\
            wall\_line\_count                     & 4              \\
            wall\_line\_width\_0                  & 0.42           \\
            xy\_offset                            & -0.05          \\
            xy\_offset\_layer\_0                  & -0.05          \\
            z\_seam\_x                            & 0              \\
            z\_seam\_y                            & 0              \\
            zig\_zaggify\_infill                  & True           \\
            \hline
        \end{tabular}
        \caption{Configurations of left extruder.}
    \end{table}
    \columnbreak
    \begin{table}[H]
        \scriptsize
        \centering
        \begin{tabular}{|l|r|}
            \hline
            \textbf{Parameter}                    & \textbf{Value} \\
            \hline
            acceleration\_topbottom               & 6000.0         \\
            acceleration\_wall\_0                 & 6000.0         \\
            acceleration\_wall\_x                 & 6000.0         \\
            brim\_line\_count                     & 8              \\
            initial\_layer\_line\_width\_factor   & 120            \\
            jerk\_print                           & 30.0           \\
            material\_final\_print\_temperature   & 215.0          \\
            material\_flow\_layer\_0              & 70             \\
            material\_initial\_print\_temperature & 215.0          \\
            material\_print\_temperature          & 215            \\
            material\_standby\_temperature        & 215.0          \\
            minimum\_bottom\_area                 & 1              \\
            minimum\_roof\_area                   & 0              \\
            minimum\_support\_area                & 10             \\
            prime\_tower\_min\_volume             & 6              \\
            skin\_monotonic                       & True           \\
            skirt\_brim\_speed                    & 20.0           \\
            speed\_infill                         & 70             \\
            speed\_layer\_0                       & 20.0           \\
            speed\_prime\_tower                   & 70.0           \\
            speed\_support                        & 70.0           \\
            speed\_topbottom                      & 70.0           \\
            speed\_wall\_0                        & 70.0           \\
            speed\_wall\_x                        & 70.0           \\
            support\_angle                        & 45             \\
            support\_bottom\_enable               & False          \\
            support\_interface\_density           & 100            \\
            support\_interface\_enable            & False          \\
            support\_interface\_offset            & 0              \\
            support\_offset                       & 1              \\
            support\_roof\_height                 & 0.5            \\
            support\_top\_distance                & 0              \\
            support\_tree\_angle\_slow            & 20             \\
            support\_tree\_top\_rate              & 35             \\
            support\_wall\_count                  & 2              \\
            support\_xy\_distance                 & 0.2            \\
            support\_xy\_distance\_overhang       & 0.1            \\
            support\_z\_distance                  & 0.1            \\
            switch\_extruder\_retraction\_amount  & 3              \\
            switch\_extruder\_retraction\_speed   & 40.0           \\
            switch\_extruder\_retraction\_speeds  & 40             \\
            top\_bottom\_pattern\_0               & con-           \\
                                                  & centric        \\
            xy\_offset\_layer\_0                  & -0.1           \\
            zig\_zaggify\_infill                  & True           \\
            \hline
        \end{tabular}
        \caption{Configurations of right extruder.}
    \end{table}
\end{multicols}


\end{document}
