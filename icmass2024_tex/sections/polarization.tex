\subsection{Polarization Properties of Reflected Light}
Unpolarized light becomes partially polarized when reflected off a surface \cite[34]{lingUniversityPhysicsVolume2016}.
This is the reason why sunglasses are commonly polarized, as they are designed to block reflected light from surfaces such as roads or water.
Figure \ref{fig:polarized_reflection} illustrates this phenomenon where the light gets partially polarized parallel to the surface (perpendicular to the plane of incidence).
At one particular angle of incidence, the reflected light is completely polarized parallel to the surface.
This angle is called the Brewster angle and is given by \cite{BrewsterAngle2024}:

\begin{equation}
    \theta_B = \arctan{\frac{n_2}{n_1}}
\end{equation}

\begin{figure}[H]
    \centering
    \includegraphics[width=.8\linewidth]{figures/polarization/reflaction.png}
    \caption{Polarization by reflection. \cite[Figure 1.38]{lingUniversityPhysicsVolume2016}}
    \label{fig:polarized_reflection}
\end{figure}


Any polarization state can be resolved as a sum of two orthogonal linear polarizations where one is perpendicular to the plane of incidence $R_\perp$ and the other is parallel to the plane of incidence $R_\parallel$ \cite{FresnelEquations2024}.
These two components are referred to as s-polarized and p-polarized light, respectively and their reflection coefficients are given by the Fresnel equations \cite{FresnelEquations2024}:

\begin{align}
    R_\perp =         & \left|{\frac {n_{1}\cos \theta _1-n_{2}\cos \theta _2}{n_{1}\cos \theta _1+n_{2}\cos \theta _2}}\right|^{2}
                      &
    R_\parallel     = & \left|{\frac {n_{1}\cos \theta _2-n_{2}\cos \theta _1}{n_{1}\cos \theta _2+n_{2}\cos \theta _1}}\right|^{2}
\end{align}

Where  $\eta_1$ and $\eta_2$ are the refractive indices of the two media,
$\theta_i$ is the angle of incidence and $\theta_r$ is the angle of refraction.

Using the trigonometric identity $ \cos^2{\left(\theta_2 \right)} = 1- \sin^2{\left(\theta_2 \right)}$ and Snell's law $\eta_1 \sin{\left(\theta_1 \right)} = \eta_2 \sin{\left(\theta_2 \right)}$ the angle of refraction, $\theta_2$ can be removed, and the equations can be written as:

\begin{align}
    R_\perp =         & \left|{\frac {n_{1}\cos \theta _1-n_{2}{\sqrt {1-\left({\frac {n_{1}}{n_{2}}}\sin \theta _1\right)^{2}}}}{n_{1}\cos \theta _1+n_{2}{\sqrt {1-\left({\frac {n_{1}}{n_{2}}}\sin \theta _1\right)^{2}}}}}\right|^{2}
                      &
    R_\parallel     = & \left|{\frac {n_{1}{\sqrt {1-\left({\frac {n_{1}}{n_{2}}}\sin \theta _1\right)^{2}}}-n_{2}\cos \theta _1}{n_{1}{\sqrt {1-\left({\frac {n_{1}}{n_{2}}}\sin \theta _1\right)^{2}}}+n_{2}\cos \theta _1}}\right|^{2}
\end{align}


Inserting the refractive index of air, $n_1 = 1$, and the refractive index of water, $n_2 = 1.33$, the reflectance can be calculated and plotted as a function of the angle of incidence, $\theta_1$, alone as shown in Figure \ref{fig:brewster0}.

The \gls{dolp} is the degree to which light is polarized and can be defined as the ratio between the difference and the sum of the two components:

\begin{figure}[H]
    \centering
    \begin{subfigure}{.5\textwidth}
        \centering

        \includegraphics[width=\textwidth]{figures/pol_plots/brewster0.pdf}
        \caption{Reflectance of S and P polarized light of water.}
        \label{fig:brewster0}
    \end{subfigure}%
    \begin{subfigure}{.5\textwidth}
        \centering
        \includegraphics[width=\textwidth]{figures/pol_plots/brewster1.pdf}
        \caption{\gls{dolp} of light reflected off water.}
        \label{fig:sub2}
    \end{subfigure}
    \caption{Reflectance and \gls{dolp} of light reflected off water as a funcion of of the angle of incidence. The Brewster angle is marked with a vertical line.}
    \label{fig:test}
\end{figure}





