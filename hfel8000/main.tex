\documentclass{iopconfser}

\usepackage{float}
\usepackage{graphicx}
\usepackage{subcaption}
\usepackage[automake]{glossaries-extra}
\usepackage{ragged2e}
\usepackage[export]{adjustbox}
\usepackage{mathtools}
\usepackage{setspace}
\onehalfspacing
\makeglossaries

\setabbreviationstyle[acronym]{long-postshort-user}
\glssetcategoryattribute{acronym}{nohyperfirst}{true}
\setabbreviationstyle{short-nolong}


% --------------------
% ---- Glossaries ----
% --------------------
\newglossaryentry{asyncio}{name=Asyncio, description={A Python library for asynchronous code.}}

% --------------------
% ----- Acronyms -----
% --------------------
\newacronym{asv}{ASV}{Autonomous Surface Vehicle}

\glsaddall
\makeglossaries

% \glsunset{cpu}
% \glsunset{gpu}
% \glsunset{lla}

% --------------------
% ----- Shortcuts ----
% --------------------


\addbibresource{mylib.bib}


\begin{document} 

\title{AI Powered Mini Ferries, the Future of Urban Transportation or Just a Hype?}

\section*{<<MilliAmpere 2>>, the new research ship from NTNU, gives us sneak peak of how bridges might become obsolete in future cities.}

\begin{figure}[H]
    \centering
    \includegraphics[width=\textwidth]{figures/milliampere.jpg}
    \caption{MilliAmpere 2 driving autonomously in Trondheim harbour \cite{hauglandDetSomHar2022}.}
\end{figure}

\section*{Artificial Intelligence in Autonomous Vehicles}
The word AI gets thrown around a lot these days, so lets clarify what we mean by AI in this context.
As an experienced driver you generally don't have to think a lot when driving a car.
You just have to stay in you lane, keep a safe distance to other vehicles and follow the speed limit.
Most of these things can already be done by a modern car as well, with adaptive cruise control and lane keeping assist.
These things are not considered as AI but rather as advanced automation, where a human has defined the rules for what to do.

When something unexpected happens however, you as a driver have to react quickly and make the right decision.
This ability to react to a new and unexpected situations is what is currently missing in autonomous vehicles and is what we refer to as AI when talking about autonomous vehicles.
The problem can be broken into three parts; 1) realizing that something unexpected is happening 2) understanding what that is 3) deciding what to do about it.


\section*{Ferries VS Cars}
Autonomous cars is something that appears to always just be a couple of years away, yet never seems to arrive.
Billions of dollars have been invested into their development but the technology is still not ready for widespread use. 
Why is the development of autonomous ferries any different?

When we start to deploy autonomous vehicles, ehther cars or ferries, there will be a transition period where a human operator is monitoring the vehicle and can take control if needed. 
For autonomous cars, the operator needs to be behind the wheel at all times to be able to react quikly enough and avoid accidents.
Ferries however are a different story.
They move significantly slower than cars, have better overview, cannot cause congestions and do not have to deal with pedestrians in the same way as road vehicles.
An autonomous ferry therefore really only needs to be able to detect that something unexpected is happening, stand still and alert a remote operator and wait for them to take control.
Being stuck on a ferry waiting for a remote operator to take control might be annoying, but it is not dangerous.
This makes the transition to fully autonomous ferries much more gradual and less risky than for cars.

\section*{MilliAmpere 2}
MilliAmpere 2 is a the new research ship from NTNU that is currently being used in the development of autonomous ferries.
With eight cameras, multiple lidars and a radar it is able to observe its surrounding and know how far away other objects are.
Researchers have developed systems that use this data and makes MilliAmpere 2 able to navigate across the harbour autonomously, avoiding other boats.
Currently its behaviour is quite conservative and it is configured to stop when it detects that something unexpected is hapening.
While the ferry can be controlled remotely by a human operator is still always present on board to take control if needed.







% \printglossary
\printbibliography


\end{document}