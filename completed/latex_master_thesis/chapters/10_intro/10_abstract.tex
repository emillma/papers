\chapter*{Abstract}

Data-driven models have emerged as a promising area of research in the field of \glspl{asv}.
These models have the potential to interpret sensor data and enhance situational awareness.
However, their effectiveness heavily relies on large quantities of high-quality data for training, testing, and validation purposes.

In this \master, I have completed the development of a human-operable \sr for collecting high-quality data in maritime environments.
The \sr is IP67 rated and has two polarization cameras, an \gls{imu}, and two \gls{gnss} receivers, all synchronized to UTC.
The \sr also exposes two additional $1Gb/s$ Ethernet ports with PoE, enabling the connection of additional sensors.

A significant technical achievement in this \master has been to leverage \gls{cuda} and GStreamer for on-the-fly hardware-accelerated video compression on the Jetson Xavier, significantly extending the recording time.
In addition, ergonomic hardware enhancements have been incorporated, along with a new web-based user interface, streamlining the data acquisition process.
Finally, several small data sets were captured in the field to demonstrate the capabilities of the sensor rig as a data collection platform.

This new low-threshold approach to acquiring high-quality data is a strong foundation for my future Ph.D. work and has garnered attention from external entities.
Orders have been placed for the necessary components to create a second \sr, and discussions with external parties for funding a third \sr are currently underway.
These developments demonstrate the positive reception of the \sr and the demand for such a system.

\chapter*{Sammendrag}
Datastyrte modeller har blitt et lovende forskningsområde innenfor autonome båter. Disse modellene har potensial til å tolke sensordata og forbedre situasjonsbevisstheten. Deres effektivitet avhenger imidlertid i stor grad av store mengder høykvalitetsdata for trening, testing og validering.

I denne masteroppgaven har jeg fullført utviklingen av en sensorrigg for innsamling av høykvalitetsdata i maritime miljøer. Sensorriggen er IP67-klassifisert og har to polariserende kameraer, en IMU og to GNSS-mottakere, alle synkronisert til UTC. Sensorriggen har også to ekstra Ethernet-porter med PoE som muliggjør tilkobling av ekstra sensorer.

En betydelig teknisk prestasjon i denne masteroppgaven har vært å utnytte CUDA og GStreamer for maskinvareakselerert videokomprimering i sanntid på Jetson Xavier, noe som betydelig utvider opptakstiden. I tillegg er det implementert ergonomiske maskinvareforbedringer, sammen med et nytt webbasert brukergrensesnitt som forenkler datainnsamlingsprosessen. Til slutt ble flere små datasett samlet inn i feltet for å demonstrere sensorriggens evner som en plattform for datainnsamling.

Denne nye tilnærmingen senker terskelen for å skaffe høykvalitets data, danner et solid grunnlag for mitt fremtidige doktorgradsarbeid, og har vekket interesse fra eksterne aktører. Det er blitt bestilt komponenter for å lage en til sensorrigg, og diskusjoner med eksterne parter om finansiering av en tredje sensorrigg er underveis. Disse utviklingene viser den positive mottakelsen av sensorriggen og etterspørselen etter et slikt system.