

\section{Polarization Cameras in the Maritime Domain}
Light behaves as a transverse wave, with its electromagnetic field oscillating in a plane perpendicular to the direction in which it travels.
Similar to how color filters are used to separate different oscillation frequencies, i.e. colors, polarization filters can be used to separate different oscillation directions, i.e. polarizations of light.
This gives us a new set of tools to analyze the scene, which is particularly useful in the maritime domain where the water surface is a reflective surface with a distinct polarization signature.
\begin{figure}[H]
    \centering
    \includegraphics[width=.8\linewidth]{figures/polarization/reflaction.png}
    \caption{Polarization by reflection.
        \cite[Figure 1.38]{lingUniversityPhysicsVolume2016}}
    \label{fig:polarized_reflection}
\end{figure}
\subsection{Polarization Properties of Reflected Light}
When light reflects off a surface, its polarization changes \cite[34]{lingUniversityPhysicsVolume2016}.
This effect is illustrated in Figure \ref{fig:polarized_reflection} where the unpolarized light reflected off a surface becomes partially polarized.
The incoming light can be described as a sum of two orthogonal linear polarizations $R_\perp$ and $R_\parallel$, where $R_\perp$ is perpendicular to the plane of incidence and $R_\parallel$ is parallel to the plane of incidence.
These two components are referred to as s-polarized and p-polarized light, respectively, and their reflection coefficients are given by the Fresnel equations below:

\begin{align}
    R_\perp =     & \left|{\frac {n_{1}\cos \theta _1-n_{2}\cos \theta _2}{n_{1}\cos \theta _1+n_{2}\cos \theta _2}}\right|^{2}
                  & 
    R_\parallel = & \left|{\frac {n_{1}\cos \theta _2-n_{2}\cos \theta _1}{n_{1}\cos \theta _2+n_{2}\cos \theta _1}}\right|^{2}
\end{align}

where $\eta_1$ and $\eta_2$ are the refractive indices of the two media,
$\theta_i$ is the angle of incidence and $\theta_r$ is the angle of refraction.
Using the trigonometric identity $ \cos^2{\left(\theta_2 \right)} = 1- \sin^2{\left(\theta_2 \right)}$ and Snell's law $\eta_1 \sin{\left(\theta_1 \right)} = \eta_2 \sin{\left(\theta_2 \right)}$ the angle of refraction, $\theta_2$ can be removed, and the equations can be rewritten as:

\begin{align}
    R_\perp =     & \left|{\frac {n_{1}\cos \theta _1-n_{2}{\sqrt {1-\left({\frac {n_{1}}{n_{2}}}\sin \theta _1\right)^{2}}}}{n_{1}\cos \theta _1+n_{2}{\sqrt {1-\left({\frac {n_{1}}{n_{2}}}\sin \theta _1\right)^{2}}}}}\right|^{2}
                  & 
    R_\parallel = & \left|{\frac {n_{1}{\sqrt {1-\left({\frac {n_{1}}{n_{2}}}\sin \theta _1\right)^{2}}}-n_{2}\cos \theta _1}{n_{1}{\sqrt {1-\left({\frac {n_{1}}{n_{2}}}\sin \theta _1\right)^{2}}}+n_{2}\cos \theta _1}}\right|^{2}
\end{align}


Inserting the refractive index of air, $n_1 = 1$, and the refractive index of water, $n_2 = 1.33$, the reflectance can be calculated and plotted as a function of the angle of incidence, $\theta_1$, alone as shown in Figure \ref{fig:brewster0}.
The \gls{dolp} is the degree to which light is linearly polarized and can be defined as the ratio between the difference and the sum of the two polarization components and is plotted in Figure \ref{fig:dolp_graph}.
At one particular angle of incidence, $\theta_B$, known as the Brewster angle, $R_\parallel$ becomes zero, and the reflected light is completely s-polarized:
\begin{align}
    DoLP= & \frac{\left | R_\perp - R_\parallel \right |}{R_\perp + R_\parallel} & \theta_B & = \arctan{\frac{n_2}{n_1}}
\end{align}

\begin{figure}[H]
    \centering
    \begin{subfigure}{.5\textwidth}
        \centering
        \includegraphics[width=\textwidth]{figures/pol_plots/brewster0.pdf}
        \caption{Reflectance of S and P polarized light of water.}
        \label{fig:brewster0}
    \end{subfigure}%
    \begin{subfigure}{.5\textwidth}
        \centering
        \includegraphics[width=\textwidth]{figures/pol_plots/brewster1.pdf}
        \caption{\gls{dolp} of light reflected off water. \label{fig:dolp_graph}}
        \label{fig:brewster1}
    \end{subfigure}
    \caption{Reflectance and \gls{dolp} of light reflected off water as a function of the angle of incidence.
        The Brewster angle is marked with a vertical line.}
    \label{fig:test}
\end{figure}

\subsection{Estimating Polarization Properties of Light using Polarizers}
In polarized light, the electromagnetic oscillations trace an elliptical path, which simplifies to a circular or straight line for pure circular or linear polarization, respectively.
When light passes through a linear polarizer, the electric field is filtered, allowing only the component aligned with the polarizer to continue through.
By aligning four linear polarizers at 45-degree intervals, the \glsfirst{dolp} and \glsfirst{aolp} can be determined as shown in Figure \ref{fig:polarization_calculation}.

\begin{figure}[H]
    
    \begin{minipage}{.48\textwidth}
        \includegraphics[width=\textwidth]{figures/polarization_sketch.pdf}
    \end{minipage}
    \hfill
    \begin{minipage}{.48\textwidth}
        \begin{alignat}{3}
             & \mathrlap{\cos\theta \cdot a\cos t -  \sin\theta \cdot b\sin t}                & 
             &                                                                                & 
             &                                                                                   \\
             &                                                                                & 
             & = \mathrlap{\sqrt{a^2\cos^2\theta+b^2\sin^2\theta} \cos(\omega + \phi)}        & 
             &                                                                                   \\[1em]
             & I_0                                                                            & 
             & = \mathrlap{a^2\cos^2\theta + b^2\sin^2\theta                                } & 
             &                                                                                   \\
             & I_{45}                                                                         & 
             & = \mathrlap{a^2\cos^2(\theta-\frac{\pi}{4}) + b^2\sin^2(\theta-\frac{\pi}{4})} & 
             &                                                                                   \\
             & I_{90}                                                                         & 
             & = \mathrlap{a^2\sin^2\theta + b^2\cos^2\theta                                } & 
             &                                                                                   \\
             & I_{135}                                                                        & 
             & = \mathrlap{a^2\sin^2(\theta-\frac{\pi}{4}) + b^2\cos^2(\theta-\frac{\pi}{4})} & 
             &                                                                                   \\[1em]
             & S_0                                                                            & 
             & = I_0 + I_{90}                                                                 & 
             & = a^2+b^2                                                                         \\
             & S_1                                                                            & 
             & = I_0 - I_{90}                                                                 & 
             & =(a^2-b^2)\cos(2x)                                                                \\
             & S_2                                                                            & 
             & = I_{45} - I_{135}                                                             & 
             & =(a^2-b^2)\sin(2x)                                                                \\[1em]
             & DoLP                                                                           & 
             & =\frac{a^2-b^2}{a^2+b^2}                                                       & 
             & = \frac{\sqrt{S_1^2 + S_2^2}}{S_0}                                                \\
             & AoLP                                                                           & 
             & =  \theta                                                                      & 
             & = \frac{1}{2}\arctan{\left(\frac{S_2}{S_1}\right)}
        \end{alignat}
    \end{minipage}%
    
    \caption{How four linear polarizers placed at $45^\circ$ intervals can be used to calculate the \gls{dolp} and \gls{aolp} of polarized light. \label{fig:polarization_calculation}}
\end{figure}%

\section{Polarization Cameras}
The sensor rig is equipped with two TRI050S1-QC cameras from Lucid Vision Labs.
The cameras use a 5MP \gls{cpfa} sensor from Sony capable of capturing both color and polarization information. 
In this sensor, every pixel is equipped with a color filter and a polarization filter, as shown in Figure \ref{fig:cpfa}.
\begin{figure}[H]
    \begin{subfigure}[B]{.48\textwidth}
        \centering
        \includegraphics[width=\textwidth]{figures/sensor_layout.pdf}
        \caption{\gls{cpfa}. The numbers represent the filter angles.\label{fig:cpfa}}
    \end{subfigure}
    \hfill
    \begin{subfigure}[B]{.48\textwidth}
        \includegraphics[width=\textwidth]{figures/sensor_packaging.pdf}
        \caption{Reordered view of \ref{fig:cpfa}. \label{fig:cpfa_reorder}}
    \end{subfigure}
    \caption{A $12\times12$ slice of the \gls{cpfa} used in TRI050S1-QC cameras \cite{lucidvisionlabsTritonMPPolarized2020}, and how the the data can be reorderd as a $3\times3$ image with 16 channels. \label{fig:polarization_sensor}}
\end{figure}%

\subsection{Color-Polarization Demosaicking}
Similar to how debayering is used to estimate the missing color channels from a Bayer filter, color-polarization demosaicking is used to estimate the missing polarization channels from a \gls{cpfa} sensor.
Due to the physical displacement of the filters on the sensors, it's important to compensate for gradients in the scene to avoid aliasing artifacts.
Several methods that solve this and perform accurate color-polarization demosaicking exist \cite{morimatsuMonochromeColorPolarization2020,morimatsuMonochromeColorPolarization2021,nguyenTwoStepColorPolarizationDemosaicking2022a}.
Still, for simplicity, we propose a linear method we implemented using a single convolutional layer on the GPU.
This method is simple, fast and yields images with few aliasing artifacts.
The method estimates $S0$, $S1$, and $S2$ for each color channel at every pixel, i.e. it transforms the $n \times m \times 1$ raw \gls{cpfa} image into a $n \times m \times 9$ image.

First, the raw \gls{cpfa} input image is permuted into a $(n/4) \times (m/4) \times 16$ image, where the 16 channels correspond to the different color and polarization filters as depicted in Figure \ref{fig:cpfa_reorder}.
A custom CUDA kernel was written to unpack raw camera data and perform this permutation in one step, but the same result can be achieved with the following tensor operations using a library like PyTorch:
\begin{figure}[H]
    \centering
    \includegraphics[width=.6\textwidth]{figures/transformation.pdf}
    \caption{Permutations from a $n \times m \times C $ tensor to a $(n/4) \times (m/4) \times (C\times 16)$ tensor. \label{fig:reorder_operations}}
    
\end{figure}%

Then, a single convolutional layer is applied to obtain a $(n/4)*(m/4)*144$ image.
Finally, the resulting image is permuted back to a $n \times m \times 9$ image by reversing the operations in Figure \ref{fig:reorder_operations}.
With kernel size 5, the convolutional layer has only $144\times16\times5\times5=57600$ weights.
Since we only use a single convolution operation, the transformation is linear, and we can find the least square solution directly without using gradient descent.
The weights were fitted to the Tokyo Tech dataset, a 12-channel (3 colors $\times$ 4 angles) color-polarization image dataset with 40 scenes \cite{morimatsuMonochromeColorPolarization2020,morimatsuMonochromeColorPolarization2021}.
To simulate raw \gls{cpfa} images, we select the channel for each pixel with the same color and polarization angle as the corresponding pixel in the \gls{cpfa} image.
Figure \ref{fig:cpfa_demosaicking} shows how this method is used to obtain and visualize the polarization information from a \gls{cpfa} image.

To increase the size of the dataset, we used all 16 possible placements of the virtual \gls{cpfa} on each image by shifting the images up to 3 pixels in each direction.
We also augmented the dataset by flipping the images horizontally and vertically.


\begin{figure}[H]
    \begin{subfigure}[T]{.49\textwidth}
        \includegraphics[width=\textwidth]{figures/img_0080_right_inten.jpg}
        \caption{$I90$, $I45$, $I135$, and $I0$.}
    \end{subfigure}
    \hfill
    \begin{subfigure}[T]{.49\textwidth}
        \includegraphics[width=\textwidth]{figures/img_0080_right_s0.jpg}
        \caption{$S_0$, equivalent to what a normal camera would capture.}
    \end{subfigure}
\end{figure}

\begin{figure}[H]\ContinuedFloat
    \begin{subfigure}[B]{.49\textwidth}
        \includegraphics[width=\textwidth]{figures/img_0080_right_pol.jpg}
        \caption{Visualization of \gls{dolp} and \gls{aolp} for luminance. Similar visualizations can be made for each separate color channel.}
    \end{subfigure}
    \hfill
    \begin{subfigure}[B]{.49\textwidth}
        \centering
        \includegraphics[width=.8\textwidth]{figures/cmap/aolp_dolp_cmap.pdf}
        \vspace{1em}
        \caption{Colormap used to visualize \gls{dolp} and \gls{aolp}. Using a colormap with constant luminance, rather than HSV, makes it easier to determine \gls{dolp} visually. }
    \end{subfigure}
    \caption{An image captured with a color polarization filter array sensor and visualization of its polarization information. \label{fig:cpfa_demosaicking}}
\end{figure}

% \hfill



% \section{Debayering of Color Polarization Images}

% \begin{align*}
%     \text{BT709} = \begin{bmatrix}
%                        0.2126  & 0.7152  & 0.0722  \\
%                        -0.1146 & -0.3854 & 0.5     \\
%                        0.5     & -0.4542 & -0.0458
%                    \end{bmatrix}
% \end{align*}


