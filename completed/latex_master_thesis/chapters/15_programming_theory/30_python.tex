\section{Concurrent programming in Python}
\label{sec:concurrent_programming_in_python}
Most of the code in this thesis is written in Python, and therefore it is important to understand how concurrent programming is implemented in \py.
There are many different ways to achieve concurrency in Python, including threading, multiprocessing, subprocesses, and \gls{asyncio}.

\subsection{Threading}
The benefits of using threads include their ease of use and seamless integration with existing code. However, CPython, the most common implementation of Python, employs the \gls{gil} as a crucial design choice. The \gls{gil} ensures that multiple threads cannot execute Python code simultaneously, unlike in most other programming languages \cite{ajitsariaWhatPythonGlobal2018}. Guido van Rossum, the creator of Python, provides an explanation for the inclusion of the GIL in Python during an episode of Lex Fridman's podcast \cite{lexfridmanGuidoVanRossum2022}..

\subsection{Multiprocessing}
In contrast to threads, multiprocessing allows for true parallelism, as each process has its own \gls{gil} \cite{ajitsariaWhatPythonGlobal2018}.
However, communicating between processes is more difficult than communicating between threads, as processes do not share memory, and data has to be serialized in order to be transferred, imposing a significant overhead \cite{pythonsoftwarefoundationMultiprocessingProcessbasedParallelism}.

\subsection{Subprocessing}
Python's \code{subprocess} module makes it possible to launch, monitor and interact with external processes \cite{pythonsoftwarefoundationSubprocessSubprocessManagement}.
The main advantage of using subprocesses is that any existing program can easily be used, regardless of the programming language it is written in.

\subsection{AsyncIO}
The \py standard library provides the \gls{asyncio} module, enabling the use of an event loop for scheduling and executing asynchronous tasks \cite{pythonsoftwarefoundationAsyncioAsynchronous}.
While \gls{asyncio} offers greater control over program execution, it requires careful program design to prevent blocking the event loop.

To use \gls{asyncio}, the program must be written in a way that allows the event loop to switch between tasks, making it difficult to use it with existing code that is not designed for asynchronous execution.
Fortunately, \gls{asyncio} makes it easy to integrate existing blocking code by running it in a thread pool, a process pool, or even as a subprocess and wrapping it in a coroutine \cite{pythonsoftwarefoundationAsyncioAsynchronous}.

Compared to threads, \gls{asyncio} is more lightweight and has less overhead, as shown in Listing \ref{listing:concurrency_test} and \ref{listing:concurrency_test_results}.
In the test, four different ways to achieve concurrency are compared, \gls{asyncio}, threads, a thread pool, and a thread pool running in \gls{asyncio}.
In the test, 10000 tasks are executed, each of which sleeps for one second.

\begin{listing}[H]
    \begin{minted}{python}
        N=10000
        def sleep_1sec(): time.sleep(1)
        async def sleep_1sec_async(): await asyncio.sleep(1)

        async def test_asyncio():
            await asyncio.gather(*[sleep_1sec_async() for _ in range(N)])

        def test_thread_pool():
            with ThreadPoolExecutor(max_workers=N) as ex:
                futures = [ex.submit(sleep_1sec) for _ in range(N)]

        async def test_thread_pool_async():
            loop = asyncio.get_event_loop()
            with ThreadPoolExecutor(max_workers=N) as ex:
                asyncio.gather(*[loop.run_in_executor(ex, sleep_1sec) for _ in range(N)])

        def test_threads():
            threads = [threading.Thread(target=do_nothing) for _ in range(N)]
            [thread.start() for thread in threads]
            [thread.join() for thread in threads]
    \end{minted}
    \caption{Code used to compatre asyncio and threads.}
    \label{listing:concurrency_test}
\end{listing}
\begin{listing}[H]
    \begin{minted}{bash}
        >>> test_asyncio took 1.373797 seconds to complete.
        >>> test_thread_pool took 4.091199 seconds to complete.
        >>> test_thread_pool_async took 4.391777 seconds to complete.
        >>> test_threads took 5.373108 seconds to complete.
    \end{minted}
    \caption{Results when running the code in Listing  \ref{listing:concurrency_test} on the \jx}
    \label{listing:concurrency_test_results}
\end{listing}

