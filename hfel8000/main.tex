\documentclass{iopconfser}

\usepackage{float}
\usepackage{graphicx}
\usepackage{subcaption}
\usepackage[automake]{glossaries-extra}
\usepackage{ragged2e}
\usepackage[export]{adjustbox}
\usepackage{mathtools}
\usepackage{setspace}
\onehalfspacing
\makeglossaries

\setabbreviationstyle[acronym]{long-postshort-user}
\glssetcategoryattribute{acronym}{nohyperfirst}{true}
\setabbreviationstyle{short-nolong}


% --------------------
% ---- Glossaries ----
% --------------------
\newglossaryentry{asyncio}{name=Asyncio, description={A Python library for asynchronous code.}}

% --------------------
% ----- Acronyms -----
% --------------------
\newacronym{asv}{ASV}{Autonomous Surface Vehicle}

\glsaddall
\makeglossaries

% \glsunset{cpu}
% \glsunset{gpu}
% \glsunset{lla}

% --------------------
% ----- Shortcuts ----
% --------------------


\addbibresource{mylib.bib}
 

\begin{document} 

\title{AI Powered Mini Ferries, the Future of Urban Transportation or Yet Another Hype?}

\section*{<<MilliAmpere 2>>, the new research ship from NTNU, gives us sneak peak of how bridges might become obsolete in future cities.}

\begin{figure}[H]
    \centering
    \includegraphics[width=\textwidth]{figures/milliampere.jpg}
    \caption{MilliAmpere 2 driving autonomously in Trondheim harbour \cite{hauglandDetSomHar2022}.}
\end{figure}

\section*{Artificial Intelligence in Autonomous Vehicles}
The word AI gets thrown around a lot these days, so lets clarify what we mean by AI in this context.
As an experienced driver you generally don't have to think a lot when driving a car.
You just have to stay in you lane, keep a safe distance to other vehicles and follow the speed limit.
Most of these things can already be done by a modern car as well, with adaptive cruise control and lane keeping assist.
The ability do this would not be called AI but is rather advanced automation, where a human has defined the rules for what to do.

When something unexpected happens however, you as a driver have to react quickly and make the right decision.
This ability to react to a new and unexpected situations is what is currently missing in autonomous vehicles and is what we refer to as AI when talking about autonomous vehicles.
The problem can be broken into three parts; 1) realizing that something unexpected is happening 2) understanding what that is 3) deciding what to do about it.


\section*{Ferries VS Cars}
Autonomous cars is something that appears to always just be a couple of years away, yet never seems to arrive.
Billions of dollars have been invested into their development but the technology is still not ready for widespread use. 
Why is the development of autonomous ferries any different?

When we start to deploy autonomous vehicles, ehther cars or ferries, there will be a transition period where a human operator is monitoring each vehicle and can take control if needed. 
For autonomous cars, the operator needs to be behind the wheel at all times to be able to react quikly enough and avoid accidents.
Ferries however are a different story.
They move significantly slower than cars, have better overview, cannot cause congestions and do not have to deal with pedestrians in the same way as road vehicles.
An autonomous ferry therefore really only needs to be able to detect that something unexpected is happening, stand still and alert a remote operator and wait for them to take control.
Being a stuck passenger on a ferry waiting for a remote operator to take control might be annoying, but it is not dangerous.
This makes the transition to fully autonomous ferries much more gradual and less risky than for cars.
Being located on the water with line of sight to infrastructure on land also makes it easier to deploy 5G networks that can be used to reliably operate the ferries remotely.

\section*{MilliAmpere 2}
MilliAmpere 2 is a the new research ship from NTNU that is currently being used in the development of autonomous ferries.
With eight cameras, multiple lidars and a radar it is able to observe its surrounding and know how far away other objects are.
Researchers have developed systems that use this data and makes MilliAmpere 2 able to navigate across the harbour autonomously, avoiding other boats.
Currently its behaviour is quite conservative and it is configured to stop when it detects that something unexpected is hapening.
Even though the ferry can be controlled remotely, a human operator is still always present on board to take control if needed.
Citizens of Trondheim have already been able to take rides across the harbour with MilliAmpere 2, and if everything goes according to plan, the ferry will be able to operate fully autonomously in the future.

It is still the early days for autonomous ferries, but the technology is rapidly improving. 
Wheiter or not the optimism around autonomous ferries will play out like the hype around autonomous cars remains to be seen, but it might appear like the bar is set significantly lower for ferries than for autonomous cars.


\newpage
\title{Reflections}
This text is aimed at the part of the general public that is slightly sceptical towards autonomy, with a goal of convincing them that autonomous ferries might be an easier problem to solve than autonomous cars.

\section*{Me the Writer}
Acknowledging my own position in relation to the topic is as a good place to start before analyzing the target audience.
As I am taking a PhD in autonomous systems I'm hopefully more knowledgeable about the technology than the average person, and should therefore be aware of the vocabulary and concepts I use to explain the technology.
While researchers might be very pedantic about details and technicalities, it is important to note that in order to engage with a broader audience it is often better to avoid the technicalities and focus on the message \cite{kulykPeopleWantReassurance2023}.

As a researcher in the field I'm probably more positive towards autonomous technology than the average person, but knowing how hard it is to implement simingly simple features makes me believe that the technology is still quite far away from being ready for widespread use.
I'm certain that the technology will benefit society, given that it is implemented correctly and with the right political regulations in place to avoid exessive increase in social inequality.

\section*{The Target Audience}
With AI and autonomy being a very hot topic in the news, I think most people are familiar with the concept and have some kind of opinion on the topic.

In terms of competence, the text is aimed at people who are familiar with the concept of autonomous vehicles, but not experienced in the field.
The text tries to avoud the use of difficult language, and the technical terms used like "lidar" are not necessary to understand in order to get the message of the text.
I assumes that the reader is familiar with driving a car, and use several comparisons and methaphors that are related to driving.
This can be an effective tool in communicating ideas as we can leverage the readers knowledge and sentiment from another topic \cite{gustafssonCognitiveLinguisticsScience2024}.
Young people and paople without a drivers license might therefore not be the best target audience for this text.

Convincing sceptics is hard and convincing optimists is not necessary, so the text is aimed at the people in the middle who are sceptical towards the technology but not completely against it.
I start by framing it as an open question to make the reader curious about the topic and avoid scaring of people who are sceptical towards the technology.


Focusing on it's use in killer robots might generate clicks, but is not the best way if you want to convince people that the technology is safe and useful.


\begin{figure}[H]
    \centering
    \includegraphics[width=.8\textwidth]{figures/overview.pdf}
    \caption{Visualization of the target audience for this text. The targeg audience is marked in red. The occupations in the figure are mostly based on my own assumptions and are quite stereotypical.}
\end{figure}

\section*{The Boy who Cried Wolf}
For a long time been this idea that autonomous vehicles are just a couple of years away have been a bit like the boy who cried wolf, making a people naturally sceptical towards the arrival of the technology. 
It seams reasonable to assume that this scepticism about the arrival is more pronounced among the people sceptical towards the technology in general.


This includes avoiding technical jargon and using more narrative driven language to make the text more engaging, as Dahlstrom points out in his article on the topic \cite{dahlstromUsingNarrativesStorytelling2014}.

\section*{How Agency is Framed}
We often use language to describe autonomous systems that imply a level of agency that is not present in the system.
For instance it is fairly common to say that a car "decides" to turn left or that it "understands" that it should stop at a red light.
For people working with autonomous systems these "decisions" are closer to how an elevator "decides" to stop at a floor, in that they are preprogrammed and not the result of any conscious thought, while for the general public the word "decide" might imply a level of agency that is not present in the system.
With the development of more advanced language models, like 


\section*{Benefit to Society}
In this text I targeted scepticisom towards the arrival of autonomous ferries, and does not delv into the topic of whether or not the autonomous ferries is a good idea.
If it had been a longer text this would have given more attention.
It has been shown that the framing can play a significant part in how people percieve AI techonlogy and it is therefore important to be aware of this when communicating about the topic \cite{bingamanSiriShowMe2021}.
Green public city transport is something that most people can agree is a good idea, but it is sort of part of a culture clash between rural people driving diesel cars and urban people drinking soy lattes.


\pagebreak
\printbibliography

\end{document}