% \subsection{Image data formats}
% As everything else, images are encoded as a sequence of bits.
% How we structure these bits in memory is another important property of image formats and can be very important for performance.
% Normal RGB images are encoded as three 8-bit values for each pixel, one for each color channel.


% \subsubsection{Interleved vs planar vs semi-planar}
% There are three main ways to store the color channels in memory, interleved, planar and semi-planar \cite{baranYUVFormats2018}.
% In interleved formats, the color channels are stored in sequence, i.e. RGBRGBRGB.
% In planar formats, the color channels are stored in separate arrays, i.e. RRRBBBGGG.
% Semi-planar formats are a mix of the two where some channels are interleved and some are planar, i.e. RRRGBGBGB.
% The reson one might be preferred over the other is to optimize for memory locality.
% For instance, if you are performing per-bit operations, having the bits of the same color channel next to each other is beneficial.


% \begin{figure}[H]
%     \centering
%     \includegraphics[width=.8\textwidth]{figures/debayer/YUV_packaging.png}
%     \caption{Visualization of I420 and NV12, two popular YUV 4:2:0 formats \cite{baranYUVFormats2018}.}
%     \label{fig:image_packaging}
% \end{figure}

% \subsubsection{Bit depth and packaging}
% A final importnt property of image formats is the bit depth, i.e. how many bits are used to represent each pixel.
% Most images use a single byte (8 bits) for each color channel, but higher bit depths offer better color fidelity and dynamic range.
% How these bits are packaged might also vary.
% If the number of bits is not divisable by 8, like for 10-bit images, the most space efficient way to store and send them is to pack them into 8-bit bytes, i.e. 4 10-bit values are stored in 5 bytes.
% However as most \glspl{alu} do not support 10-bit operations, padding the to the neares full byte is better for computations.

